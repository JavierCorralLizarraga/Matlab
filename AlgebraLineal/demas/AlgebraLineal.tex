% This LaTeX was auto-generated from MATLAB code.
% To make changes, update the MATLAB code and export to LaTeX again.

\documentclass{article}

\usepackage[utf8]{inputenc}
\usepackage[T1]{fontenc}
\usepackage{lmodern}
\usepackage{graphicx}
\usepackage{color}
\usepackage{listings}
\usepackage{hyperref}
\usepackage{amsmath}
\usepackage{amsfonts}
\usepackage{epstopdf}
\usepackage[table]{xcolor}
\usepackage{matlab}

\sloppy
\epstopdfsetup{outdir=./}
\graphicspath{ {./AlgebraLineal_images/} }

\begin{document}

\label{T_89FDB9C5}
\matlabtitle{Algebra Lineal}


\label{H_37F35C54}
\matlabheadingthree{Ejemplos de manejar matrices}

\begin{par}
\begin{flushleft}
\includegraphics[width=\maxwidth{40.642247867536376em}]{image_0}\includegraphics[width=\maxwidth{41.54540893125941em}]{image_1}
\end{flushleft}
\end{par}


\begin{par}
\begin{flushleft}
entendiendo la multiplicacion
\end{flushleft}
\end{par}

\begin{matlabsymbolicoutput}
C2 = 
    $\displaystyle \left(\begin{array}{ccc}
b_{1,1} \,\bar{a_{1,1} } +b_{2,1} \,\bar{a_{1,2} } +b_{3,1} \,\bar{a_{1,3} }  & b_{1,2} \,\bar{a_{1,1} } +b_{2,2} \,\bar{a_{1,2} } +b_{3,2} \,\bar{a_{1,3} }  & b_{1,3} \,\bar{a_{1,1} } +b_{2,3} \,\bar{a_{1,2} } +b_{3,3} \,\bar{a_{1,3} } \\
b_{1,1} \,\bar{a_{2,1} } +b_{2,1} \,\bar{a_{2,2} } +b_{3,1} \,\bar{a_{2,3} }  & b_{1,2} \,\bar{a_{2,1} } +b_{2,2} \,\bar{a_{2,2} } +b_{3,2} \,\bar{a_{2,3} }  & b_{1,3} \,\bar{a_{2,1} } +b_{2,3} \,\bar{a_{2,2} } +b_{3,3} \,\bar{a_{2,3} } \\
b_{1,1} \,\bar{a_{3,1} } +b_{2,1} \,\bar{a_{3,2} } +b_{3,1} \,\bar{a_{3,3} }  & b_{1,2} \,\bar{a_{3,1} } +b_{2,2} \,\bar{a_{3,2} } +b_{3,2} \,\bar{a_{3,3} }  & b_{1,3} \,\bar{a_{3,1} } +b_{2,3} \,\bar{a_{3,2} } +b_{3,3} \,\bar{a_{3,3} } 
\end{array}\right)$
\end{matlabsymbolicoutput}
\label{H_40A108B2}
\matlabheading{Tarea 1}

\label{H_B89A859A}
\matlabheadingthree{Ej. 1}


\begin{par}
\begin{flushleft}
a)  
\end{flushleft}
\end{par}

\begin{matlabsymbolicoutput}
A = 
    $\displaystyle \left(\begin{array}{ccccc}
2 & -4 & 12 & -10 & 58\\
-1 & 2 & -3 & 2 & -14\\
2 & -4 & 9 & -6 & 44
\end{array}\right)$
\end{matlabsymbolicoutput}
\begin{matlabsymbolicoutput}
A1 = 
    $\displaystyle \left(\begin{array}{ccccc}
-1 & 2 & -3 & 2 & -14\\
2 & -4 & 12 & -10 & 58\\
2 & -4 & 9 & -6 & 44
\end{array}\right)$
\end{matlabsymbolicoutput}
\begin{matlabsymbolicoutput}
A2 = 
    $\displaystyle \left(\begin{array}{ccccc}
1 & -2 & 3 & -2 & 14\\
2 & -4 & 12 & -10 & 58\\
2 & -4 & 9 & -6 & 44
\end{array}\right)$
\end{matlabsymbolicoutput}
\begin{matlabsymbolicoutput}
A3 = 
    $\displaystyle \left(\begin{array}{ccccc}
1 & -2 & 3 & -2 & 14\\
0 & 0 & 6 & -6 & 30\\
2 & -4 & 9 & -6 & 44
\end{array}\right)$
\end{matlabsymbolicoutput}
\begin{matlabsymbolicoutput}
A4 = 
    $\displaystyle \left(\begin{array}{ccccc}
1 & -2 & 3 & -2 & 14\\
0 & 0 & 6 & -6 & 30\\
0 & 0 & 3 & -2 & 16
\end{array}\right)$
\end{matlabsymbolicoutput}
\begin{matlabsymbolicoutput}
A5 = 
    $\displaystyle \left(\begin{array}{ccccc}
1 & -2 & 3 & -2 & 14\\
0 & 0 & 0 & -2 & -2\\
0 & 0 & 3 & -2 & 16
\end{array}\right)$
\end{matlabsymbolicoutput}
\begin{matlabsymbolicoutput}
A6 = 
    $\displaystyle \left(\begin{array}{ccccc}
1 & -2 & 3 & -2 & 14\\
0 & 0 & 3 & -2 & 16\\
0 & 0 & 0 & -2 & -2
\end{array}\right)$
\end{matlabsymbolicoutput}
\begin{matlabsymbolicoutput}
A7 = 
    $\displaystyle \left(\begin{array}{ccccc}
1 & -2 & 3 & -2 & 14\\
0 & 0 & 3 & 0 & 18\\
0 & 0 & 0 & -2 & -2
\end{array}\right)$
\end{matlabsymbolicoutput}
\begin{matlabsymbolicoutput}
A8 = 
    $\displaystyle \left(\begin{array}{ccccc}
1 & -2 & 3 & 0 & 16\\
0 & 0 & 3 & 0 & 18\\
0 & 0 & 0 & -2 & -2
\end{array}\right)$
\end{matlabsymbolicoutput}
\begin{matlabsymbolicoutput}
A9 = 
    $\displaystyle \left(\begin{array}{ccccc}
1 & -2 & 0 & 0 & -2\\
0 & 0 & 3 & 0 & 18\\
0 & 0 & 0 & -2 & -2
\end{array}\right)$
\end{matlabsymbolicoutput}
\begin{matlabsymbolicoutput}
A10 = 
    $\displaystyle \left(\begin{array}{ccccc}
1 & -2 & 0 & 0 & -2\\
0 & 0 & 1 & 0 & 6\\
0 & 0 & 0 & -2 & -2
\end{array}\right)$
\end{matlabsymbolicoutput}
\begin{matlabsymbolicoutput}
A11 = 
    $\displaystyle \left(\begin{array}{ccccc}
1 & -2 & 0 & 0 & -2\\
0 & 0 & 1 & 0 & 6\\
0 & 0 & 0 & 1 & 1
\end{array}\right)$
\end{matlabsymbolicoutput}
\begin{matlabsymbolicoutput}
ans = 
    $\displaystyle \left(\begin{array}{ccccc}
1 & -2 & 0 & 0 & -2\\
0 & 0 & 1 & 0 & 6\\
0 & 0 & 0 & 1 & 1
\end{array}\right)$
\end{matlabsymbolicoutput}
\begin{par}
\begin{flushleft}
queda una ecuacion con 2 incognitas (not sure que pedo) 
\end{flushleft}
\end{par}

\begin{par}
\begin{flushleft}
comprobe qeu estuviera bien con rref
\end{flushleft}
\end{par}


\begin{par}
\begin{flushleft}
b)
\end{flushleft}
\end{par}

\begin{matlabsymbolicoutput}
A = 
    $\displaystyle \left(\begin{array}{cccc}
3 & -3 & 3 & 9\\
2 & -1 & 4 & 7\\
3 & -5 & -1 & 7
\end{array}\right)$
\end{matlabsymbolicoutput}
\begin{matlabsymbolicoutput}
A1 = 
    $\displaystyle \left(\begin{array}{cccc}
1 & -1 & 1 & 3\\
2 & -1 & 4 & 7\\
3 & -5 & -1 & 7
\end{array}\right)$
\end{matlabsymbolicoutput}
\begin{matlabsymbolicoutput}
A2 = 
    $\displaystyle \left(\begin{array}{cccc}
1 & -1 & 1 & 3\\
0 & 1 & 2 & 1\\
3 & -5 & -1 & 7
\end{array}\right)$
\end{matlabsymbolicoutput}
\begin{matlabsymbolicoutput}
A3 = 
    $\displaystyle \left(\begin{array}{cccc}
1 & -1 & 1 & 3\\
0 & 1 & 2 & 1\\
0 & -2 & -4 & -2
\end{array}\right)$
\end{matlabsymbolicoutput}
\begin{matlabsymbolicoutput}
A4 = 
    $\displaystyle \left(\begin{array}{cccc}
1 & 0 & 3 & 4\\
0 & 1 & 2 & 1\\
0 & -2 & -4 & -2
\end{array}\right)$
\end{matlabsymbolicoutput}
\begin{matlabsymbolicoutput}
A5 = 
    $\displaystyle \left(\begin{array}{cccc}
1 & 0 & 3 & 4\\
0 & 1 & 2 & 1\\
0 & 0 & 0 & 0
\end{array}\right)$
\end{matlabsymbolicoutput}
\begin{matlabsymbolicoutput}
ans = 
    $\displaystyle \left(\begin{array}{cccc}
1 & 0 & 3 & 4\\
0 & 1 & 2 & 1\\
0 & 0 & 0 & 0
\end{array}\right)$
\end{matlabsymbolicoutput}
\begin{par}
\begin{flushleft}
mismo caso que el a)
\end{flushleft}
\end{par}

\begin{par}
\begin{flushleft}
correcto por rref
\end{flushleft}
\end{par}


\begin{par}
\begin{flushleft}
c)
\end{flushleft}
\end{par}

\begin{matlabsymbolicoutput}
A = 
    $\displaystyle \left(\begin{array}{cccc}
3 & -3 & 3 & 9\\
2 & -1 & 4 & 7\\
3 & -5 & -1 & 6
\end{array}\right)$
\end{matlabsymbolicoutput}
\begin{matlabsymbolicoutput}
A1 = 
    $\displaystyle \left(\begin{array}{cccc}
1 & -1 & 1 & 3\\
2 & -1 & 4 & 7\\
3 & -5 & -1 & 6
\end{array}\right)$
\end{matlabsymbolicoutput}
\begin{matlabsymbolicoutput}
A2 = 
    $\displaystyle \left(\begin{array}{cccc}
1 & -1 & 1 & 3\\
0 & 1 & 2 & 1\\
3 & -5 & -1 & 6
\end{array}\right)$
\end{matlabsymbolicoutput}
\begin{matlabsymbolicoutput}
A3 = 
    $\displaystyle \left(\begin{array}{cccc}
1 & -1 & 1 & 3\\
0 & 1 & 2 & 1\\
0 & -2 & -4 & -3
\end{array}\right)$
\end{matlabsymbolicoutput}
\begin{matlabsymbolicoutput}
A4 = 
    $\displaystyle \left(\begin{array}{cccc}
1 & 0 & 3 & 4\\
0 & 1 & 2 & 1\\
0 & -2 & -4 & -3
\end{array}\right)$
\end{matlabsymbolicoutput}
\begin{matlabsymbolicoutput}
A5 = 
    $\displaystyle \left(\begin{array}{cccc}
1 & 0 & 3 & 4\\
0 & 1 & 2 & 1\\
0 & 0 & 0 & -1
\end{array}\right)$
\end{matlabsymbolicoutput}
\begin{matlabsymbolicoutput}
A6 = 
    $\displaystyle \left(\begin{array}{cccc}
1 & 0 & 3 & 4\\
0 & 1 & 2 & 1\\
0 & 0 & 0 & 1
\end{array}\right)$
\end{matlabsymbolicoutput}
\begin{matlabsymbolicoutput}
ans = 
    $\displaystyle \left(\begin{array}{cccc}
1 & 0 & 3 & 0\\
0 & 1 & 2 & 0\\
0 & 0 & 0 & 1
\end{array}\right)$
\end{matlabsymbolicoutput}
\begin{par}
\begin{flushleft}
Contradiccion en el ultimo row por tanto no es consistente
\end{flushleft}
\end{par}

\begin{par}
\begin{flushleft}
correcto por rref
\end{flushleft}
\end{par}


\label{H_54605C4E}
\matlabheadingthree{Ej 2}

\begin{par}
\begin{flushleft}
a)
\end{flushleft}
\end{par}

\begin{matlabsymbolicoutput}
A = 
    $\displaystyle \left(\begin{array}{ccccc}
2 & 3 & 1 & 0 & 5\\
5 & 7 & -4 & 0 & 0\\
1 & 0 & 0 & -19 & -35\\
0 & 1 & 0 & 13 & 25\\
0 & 0 & 1 & -1 & 0
\end{array}\right)$
\end{matlabsymbolicoutput}
\begin{matlabsymbolicoutput}
A1 = 
    $\displaystyle \left(\begin{array}{ccccc}
0 & 3 & 1 & 38 & 75\\
5 & 7 & -4 & 0 & 0\\
1 & 0 & 0 & -19 & -35\\
0 & 1 & 0 & 13 & 25\\
0 & 0 & 1 & -1 & 0
\end{array}\right)$
\end{matlabsymbolicoutput}
\begin{matlabsymbolicoutput}
A2 = 
    $\displaystyle \left(\begin{array}{ccccc}
1 & 0 & 0 & -19 & -35\\
5 & 7 & -4 & 0 & 0\\
0 & 3 & 1 & 38 & 75\\
0 & 1 & 0 & 13 & 25\\
0 & 0 & 1 & -1 & 0
\end{array}\right)$
\end{matlabsymbolicoutput}
\begin{matlabsymbolicoutput}
A3 = 
    $\displaystyle \left(\begin{array}{ccccc}
1 & 0 & 0 & -19 & -35\\
0 & 7 & -4 & 95 & 175\\
0 & 3 & 1 & 38 & 75\\
0 & 1 & 0 & 13 & 25\\
0 & 0 & 1 & -1 & 0
\end{array}\right)$
\end{matlabsymbolicoutput}
\begin{matlabsymbolicoutput}
A4 = 
    $\displaystyle \left(\begin{array}{ccccc}
1 & 0 & 0 & -19 & -35\\
0 & 1 & 0 & 13 & 25\\
0 & 3 & 1 & 38 & 75\\
0 & 7 & -4 & 95 & 175\\
0 & 0 & 1 & -1 & 0
\end{array}\right)$
\end{matlabsymbolicoutput}
\begin{matlabsymbolicoutput}
A5 = 
    $\displaystyle \left(\begin{array}{ccccc}
1 & 0 & 0 & -19 & -35\\
0 & 1 & 0 & 13 & 25\\
0 & 0 & 1 & -1 & 0\\
0 & 7 & -4 & 95 & 175\\
0 & 3 & 1 & 38 & 75
\end{array}\right)$
\end{matlabsymbolicoutput}
\begin{matlabsymbolicoutput}
A6 = 
    $\displaystyle \left(\begin{array}{ccccc}
1 & 0 & 0 & -19 & -35\\
0 & 1 & 0 & 13 & 25\\
0 & 0 & 1 & -1 & 0\\
0 & 0 & -4 & 4 & 0\\
0 & 3 & 1 & 38 & 75
\end{array}\right)$
\end{matlabsymbolicoutput}
\begin{matlabsymbolicoutput}
A7 = 
    $\displaystyle \left(\begin{array}{ccccc}
1 & 0 & 0 & -19 & -35\\
0 & 1 & 0 & 13 & 25\\
0 & 0 & 1 & -1 & 0\\
0 & 0 & -4 & 4 & 0\\
0 & 0 & 1 & -1 & 0
\end{array}\right)$
\end{matlabsymbolicoutput}
\begin{matlabsymbolicoutput}
A8 = 
    $\displaystyle \left(\begin{array}{ccccc}
1 & 0 & 0 & -19 & -35\\
0 & 1 & 0 & 13 & 25\\
0 & 0 & 1 & -1 & 0\\
0 & 0 & 0 & 0 & 0\\
0 & 0 & 1 & -1 & 0
\end{array}\right)$
\end{matlabsymbolicoutput}
\begin{matlabsymbolicoutput}
A9 = 
    $\displaystyle \left(\begin{array}{ccccc}
1 & 0 & 0 & -19 & -35\\
0 & 1 & 0 & 13 & 25\\
0 & 0 & 1 & -1 & 0\\
0 & 0 & 0 & 0 & 0\\
0 & 0 & 0 & 0 & 0
\end{array}\right)$
\end{matlabsymbolicoutput}
\begin{matlabsymbolicoutput}
ans = 
    $\displaystyle \left(\begin{array}{ccccc}
1 & 0 & 0 & -19 & -35\\
0 & 1 & 0 & 13 & 25\\
0 & 0 & 1 & -1 & 0\\
0 & 0 & 0 & 0 & 0\\
0 & 0 & 0 & 0 & 0
\end{array}\right)$
\end{matlabsymbolicoutput}
\begin{par}
\begin{flushleft}
ya no se que mas \textbf{aqui hay algo qeu teines que corregir lo antes posible}
\end{flushleft}
\end{par}

\begin{par}
\begin{flushleft}
correcto por rref
\end{flushleft}
\end{par}


\begin{par}
\begin{flushleft}
respuesta alternativa
\end{flushleft}
\end{par}

\begin{matlabsymbolicoutput}
q1 = 
    $\displaystyle x=19\,t$
\end{matlabsymbolicoutput}
\begin{matlabsymbolicoutput}
q2 = 
    $\displaystyle y=25-14\,t$
\end{matlabsymbolicoutput}
\begin{matlabsymbolicoutput}
q3 = 
    $\displaystyle z=t$
\end{matlabsymbolicoutput}
\begin{matlabsymbolicoutput}
q4 = 
    $\displaystyle 2\,x+3\,y+z=5$
\end{matlabsymbolicoutput}
\begin{matlabsymbolicoutput}
q5 = 
    $\displaystyle 5\,x+7\,y-4\,z=0$
\end{matlabsymbolicoutput}
\begin{matlaboutput}
 
ans =
 
Empty sym: 0-by-1
 
\end{matlaboutput}
\begin{matlaboutput}
 
ans =
 
Empty sym: 0-by-1
 
\end{matlaboutput}
\begin{matlaboutput}
 
ans =
 
Empty sym: 0-by-1
 
\end{matlaboutput}
\begin{matlaboutput}
 
ans =
 
Empty sym: 0-by-1
 
\end{matlaboutput}

\begin{par}
\begin{flushleft}
b)
\end{flushleft}
\end{par}

\begin{matlabsymbolicoutput}
A = 
    $\displaystyle \left(\begin{array}{ccccccc}
2 & 5 & 9 & 3 & 0 & 0 & -1\\
1 & 2 & 4 & 0 & 0 & 0 & 1\\
1 & 0 & 0 & 0 & 0 & -19 & -35\\
0 & 1 & 0 & 0 & 1 & 0 & 0\\
0 & 0 & 0 & 1 & 0 & 1 & 0\\
0 & 0 & -1 & 0 & -1 & -3 & 3
\end{array}\right)$
\end{matlabsymbolicoutput}
\begin{matlabsymbolicoutput}
A1 = 
    $\displaystyle \left(\begin{array}{ccccccc}
1 & 0 & 0 & 0 & 0 & -19 & -35\\
1 & 2 & 4 & 0 & 0 & 0 & 1\\
2 & 5 & 9 & 3 & 0 & 0 & -1\\
0 & 1 & 0 & 0 & 1 & 0 & 0\\
0 & 0 & 0 & 1 & 0 & 1 & 0\\
0 & 0 & -1 & 0 & -1 & -3 & 3
\end{array}\right)$
\end{matlabsymbolicoutput}
\begin{matlabsymbolicoutput}
A2 = 
    $\displaystyle \left(\begin{array}{ccccccc}
1 & 0 & 0 & 0 & 0 & -19 & -35\\
0 & 2 & 4 & 0 & 0 & 19 & 36\\
2 & 5 & 9 & 3 & 0 & 0 & -1\\
0 & 1 & 0 & 0 & 1 & 0 & 0\\
0 & 0 & 0 & 1 & 0 & 1 & 0\\
0 & 0 & -1 & 0 & -1 & -3 & 3
\end{array}\right)$
\end{matlabsymbolicoutput}
\begin{matlabsymbolicoutput}
A3 = 
    $\displaystyle \left(\begin{array}{ccccccc}
1 & 0 & 0 & 0 & 0 & -19 & -35\\
0 & 2 & 4 & 0 & 0 & 19 & 36\\
0 & 5 & 9 & 3 & 0 & 38 & 69\\
0 & 1 & 0 & 0 & 1 & 0 & 0\\
0 & 0 & 0 & 1 & 0 & 1 & 0\\
0 & 0 & -1 & 0 & -1 & -3 & 3
\end{array}\right)$
\end{matlabsymbolicoutput}
\begin{matlabsymbolicoutput}
A4 = 
    $\displaystyle \left(\begin{array}{ccccccc}
1 & 0 & 0 & 0 & 0 & -19 & -35\\
0 & 1 & 0 & 0 & 1 & 0 & 0\\
0 & 5 & 9 & 3 & 0 & 38 & 69\\
0 & 2 & 4 & 0 & 0 & 19 & 36\\
0 & 0 & 0 & 1 & 0 & 1 & 0\\
0 & 0 & -1 & 0 & -1 & -3 & 3
\end{array}\right)$
\end{matlabsymbolicoutput}
\begin{matlabsymbolicoutput}
A5 = 
    $\displaystyle \left(\begin{array}{ccccccc}
1 & 0 & 0 & 0 & 0 & -19 & -35\\
0 & 1 & 0 & 0 & 1 & 0 & 0\\
0 & 0 & 9 & 3 & -5 & 38 & 69\\
0 & 2 & 4 & 0 & 0 & 19 & 36\\
0 & 0 & 0 & 1 & 0 & 1 & 0\\
0 & 0 & -1 & 0 & -1 & -3 & 3
\end{array}\right)$
\end{matlabsymbolicoutput}
\begin{matlabsymbolicoutput}
A6 = 
    $\displaystyle \left(\begin{array}{ccccccc}
1 & 0 & 0 & 0 & 0 & -19 & -35\\
0 & 1 & 0 & 0 & 1 & 0 & 0\\
0 & 0 & 9 & 3 & -5 & 38 & 69\\
0 & 0 & 4 & 0 & -2 & 19 & 36\\
0 & 0 & 0 & 1 & 0 & 1 & 0\\
0 & 0 & -1 & 0 & -1 & -3 & 3
\end{array}\right)$
\end{matlabsymbolicoutput}
\begin{matlabsymbolicoutput}
A7 = 
    $\displaystyle \left(\begin{array}{ccccccc}
1 & 0 & 0 & 0 & 0 & -19 & -35\\
0 & 1 & 0 & 0 & 1 & 0 & 0\\
0 & 0 & 9 & 3 & -5 & 38 & 69\\
0 & 0 & 4 & 0 & -2 & 19 & 36\\
0 & 0 & 0 & 1 & 0 & 1 & 0\\
0 & 0 & 1 & 0 & 1 & 3 & -3
\end{array}\right)$
\end{matlabsymbolicoutput}
\begin{matlabsymbolicoutput}
A8 = 
    $\displaystyle \left(\begin{array}{ccccccc}
1 & 0 & 0 & 0 & 0 & -19 & -35\\
0 & 1 & 0 & 0 & 1 & 0 & 0\\
0 & 0 & 1 & 0 & 1 & 3 & -3\\
0 & 0 & 4 & 0 & -2 & 19 & 36\\
0 & 0 & 0 & 1 & 0 & 1 & 0\\
0 & 0 & 9 & 3 & -5 & 38 & 69
\end{array}\right)$
\end{matlabsymbolicoutput}
\begin{matlabsymbolicoutput}
A9 = 
    $\displaystyle \left(\begin{array}{ccccccc}
1 & 0 & 0 & 0 & 0 & -19 & -35\\
0 & 1 & 0 & 0 & 1 & 0 & 0\\
0 & 0 & 1 & 0 & 1 & 3 & -3\\
0 & 0 & 0 & 0 & -6 & 7 & 48\\
0 & 0 & 0 & 1 & 0 & 1 & 0\\
0 & 0 & 9 & 3 & -5 & 38 & 69
\end{array}\right)$
\end{matlabsymbolicoutput}
\begin{matlabsymbolicoutput}
A10 = 
    $\displaystyle \left(\begin{array}{ccccccc}
1 & 0 & 0 & 0 & 0 & -19 & -35\\
0 & 1 & 0 & 0 & 1 & 0 & 0\\
0 & 0 & 1 & 0 & 1 & 3 & -3\\
0 & 0 & 0 & 0 & -6 & 7 & 48\\
0 & 0 & 0 & 1 & 0 & 1 & 0\\
0 & 0 & 0 & 3 & -14 & 11 & 96
\end{array}\right)$
\end{matlabsymbolicoutput}
\begin{matlabsymbolicoutput}
A11 = 
    $\displaystyle \left(\begin{array}{ccccccc}
1 & 0 & 0 & 0 & 0 & -19 & -35\\
0 & 1 & 0 & 0 & 1 & 0 & 0\\
0 & 0 & 1 & 0 & 1 & 3 & -3\\
0 & 0 & 0 & 1 & 0 & 1 & 0\\
0 & 0 & 0 & 0 & -6 & 7 & 48\\
0 & 0 & 0 & 3 & -14 & 11 & 96
\end{array}\right)$
\end{matlabsymbolicoutput}
\begin{matlabsymbolicoutput}
A12 = 
    $\displaystyle \left(\begin{array}{ccccccc}
1 & 0 & 0 & 0 & 0 & -19 & -35\\
0 & 1 & 0 & 0 & 1 & 0 & 0\\
0 & 0 & 1 & 0 & 1 & 3 & -3\\
0 & 0 & 0 & 1 & 0 & 1 & 0\\
0 & 0 & 0 & 0 & -6 & 7 & 48\\
0 & 0 & 0 & 0 & -14 & 8 & 96
\end{array}\right)$
\end{matlabsymbolicoutput}
\begin{matlabsymbolicoutput}
A13 = 
    $\displaystyle \left(\begin{array}{ccccccc}
1 & 0 & 0 & 0 & 0 & -19 & -35\\
0 & 1 & 0 & 0 & 1 & 0 & 0\\
0 & 0 & 1 & 0 & 1 & 3 & -3\\
0 & 0 & 0 & 1 & 0 & 1 & 0\\
0 & 0 & 0 & 0 & 1 & -\frac{7}{6} & -8\\
0 & 0 & 0 & 0 & -14 & 8 & 96
\end{array}\right)$
\end{matlabsymbolicoutput}
\begin{matlabsymbolicoutput}
A14 = 
    $\displaystyle \left(\begin{array}{ccccccc}
1 & 0 & 0 & 0 & 0 & -19 & -35\\
0 & 1 & 0 & 0 & 1 & 0 & 0\\
0 & 0 & 1 & 0 & 1 & 3 & -3\\
0 & 0 & 0 & 1 & 0 & 1 & 0\\
0 & 0 & 0 & 0 & 1 & -\frac{7}{6} & -8\\
0 & 0 & 0 & 0 & 0 & -\frac{25}{3} & -16
\end{array}\right)$
\end{matlabsymbolicoutput}
\begin{matlabsymbolicoutput}
A15 = 
    $\displaystyle \left(\begin{array}{ccccccc}
1 & 0 & 0 & 0 & 0 & -19 & -35\\
0 & 1 & 0 & 0 & 0 & \frac{7}{6} & 8\\
0 & 0 & 1 & 0 & 1 & 3 & -3\\
0 & 0 & 0 & 1 & 0 & 1 & 0\\
0 & 0 & 0 & 0 & 1 & -\frac{7}{6} & -8\\
0 & 0 & 0 & 0 & 0 & -\frac{25}{3} & -16
\end{array}\right)$
\end{matlabsymbolicoutput}
\begin{matlabsymbolicoutput}
A16 = 
    $\displaystyle \left(\begin{array}{ccccccc}
1 & 0 & 0 & 0 & 0 & -19 & -35\\
0 & 1 & 0 & 0 & 0 & \frac{7}{6} & 8\\
0 & 0 & 1 & 0 & 0 & \frac{25}{6} & 5\\
0 & 0 & 0 & 1 & 0 & 1 & 0\\
0 & 0 & 0 & 0 & 1 & -\frac{7}{6} & -8\\
0 & 0 & 0 & 0 & 0 & -\frac{25}{3} & -16
\end{array}\right)$
\end{matlabsymbolicoutput}
\begin{matlabsymbolicoutput}
ans = 
    $\displaystyle \left(\begin{array}{ccccccc}
1 & 0 & 0 & 0 & 0 & 0 & \frac{37}{25}\\
0 & 1 & 0 & 0 & 0 & 0 & \frac{144}{25}\\
0 & 0 & 1 & 0 & 0 & 0 & -3\\
0 & 0 & 0 & 1 & 0 & 0 & -\frac{48}{25}\\
0 & 0 & 0 & 0 & 1 & 0 & -\frac{144}{25}\\
0 & 0 & 0 & 0 & 0 & 1 & \frac{48}{25}
\end{array}\right)$
\end{matlabsymbolicoutput}
\begin{par}
\begin{flushleft}
correcto por rref
\end{flushleft}
\end{par}


\label{H_F5A554A1}
\matlabheadingthree{Ej 3}

\begin{par}
\hfill \break
\end{par}

\begin{matlabsymbolicoutput}
A = 
    $\displaystyle \left(\begin{array}{cccc}
1 & -2 & 4 & 12\\
2 & -1 & 5 & 18\\
-1 & 3 & -3 & -8
\end{array}\right)$
\end{matlabsymbolicoutput}
\begin{matlabsymbolicoutput}
A1 = 
    $\displaystyle \left(\begin{array}{cccc}
1 & -2 & 4 & 12\\
0 & 3 & -3 & -6\\
-1 & 3 & -3 & -8
\end{array}\right)$
\end{matlabsymbolicoutput}
\begin{matlabsymbolicoutput}
A2 = 
    $\displaystyle \left(\begin{array}{cccc}
1 & -2 & 4 & 12\\
0 & 3 & -3 & -6\\
0 & 1 & 1 & 4
\end{array}\right)$
\end{matlabsymbolicoutput}
\begin{matlabsymbolicoutput}
A3 = 
    $\displaystyle \left(\begin{array}{cccc}
1 & -2 & 4 & 12\\
0 & 1 & -1 & -2\\
0 & 1 & 1 & 4
\end{array}\right)$
\end{matlabsymbolicoutput}
\begin{matlabsymbolicoutput}
A4 = 
    $\displaystyle \left(\begin{array}{cccc}
1 & -2 & 4 & 12\\
0 & 1 & -1 & -2\\
0 & 0 & 2 & 6
\end{array}\right)$
\end{matlabsymbolicoutput}
\begin{matlabsymbolicoutput}
A5 = 
    $\displaystyle \left(\begin{array}{cccc}
1 & 0 & 2 & 8\\
0 & 1 & -1 & -2\\
0 & 0 & 2 & 6
\end{array}\right)$
\end{matlabsymbolicoutput}
\begin{matlabsymbolicoutput}
A6 = 
    $\displaystyle \left(\begin{array}{cccc}
1 & 0 & 0 & 2\\
0 & 1 & -1 & -2\\
0 & 0 & 2 & 6
\end{array}\right)$
\end{matlabsymbolicoutput}
\begin{matlabsymbolicoutput}
A7 = 
    $\displaystyle \left(\begin{array}{cccc}
1 & 0 & 0 & 2\\
0 & 1 & 0 & 1\\
0 & 0 & 2 & 6
\end{array}\right)$
\end{matlabsymbolicoutput}
\begin{matlabsymbolicoutput}
A8 = 
    $\displaystyle \left(\begin{array}{cccc}
1 & 0 & 0 & 2\\
0 & 1 & 0 & 1\\
0 & 0 & 1 & 3
\end{array}\right)$
\end{matlabsymbolicoutput}
\begin{matlabsymbolicoutput}
ans = 
    $\displaystyle \left(\begin{array}{cccc}
1 & 0 & 0 & 2\\
0 & 1 & 0 & 1\\
0 & 0 & 1 & 3
\end{array}\right)$
\end{matlabsymbolicoutput}
\begin{par}
\begin{flushleft}
correcto por rref
\end{flushleft}
\end{par}


\label{H_10531512}
\matlabheadingthree{Ej 4}

\begin{par}
\hfill \break
\end{par}

\begin{matlabsymbolicoutput}
A = 
    $\displaystyle \left(\begin{array}{cccc}
1 & 2 & -1 & a\\
2 & 1 & 3 & b\\
1 & -4 & 9 & c
\end{array}\right)$
\end{matlabsymbolicoutput}
\begin{matlabsymbolicoutput}
A1 = 
    $\displaystyle \left(\begin{array}{cccc}
1 & 2 & -1 & a\\
0 & -3 & 5 & b-2\,a\\
1 & -4 & 9 & c
\end{array}\right)$
\end{matlabsymbolicoutput}
\begin{matlabsymbolicoutput}
A2 = 
    $\displaystyle \left(\begin{array}{cccc}
1 & 2 & -1 & a\\
0 & -3 & 5 & b-2\,a\\
0 & -6 & 10 & c-a
\end{array}\right)$
\end{matlabsymbolicoutput}
\begin{matlabsymbolicoutput}
A3 = 
    $\displaystyle \left(\begin{array}{cccc}
1 & 2 & -1 & a\\
0 & -3 & 5 & b-2\,a\\
0 & 0 & 0 & 3\,a-2\,b+c
\end{array}\right)$
\end{matlabsymbolicoutput}
\begin{matlabsymbolicoutput}
A4 = 
    $\displaystyle \left(\begin{array}{cccc}
1 & 2 & -1 & a\\
0 & 1 & -\frac{5}{3} & \frac{2\,a}{3}-\frac{b}{3}\\
0 & 0 & 0 & 3\,a-2\,b+c
\end{array}\right)$
\end{matlabsymbolicoutput}
\begin{matlabsymbolicoutput}
A5 = 
    $\displaystyle \left(\begin{array}{cccc}
1 & 0 & \frac{7}{3} & \frac{2\,b}{3}-\frac{a}{3}\\
0 & 1 & -\frac{5}{3} & \frac{2\,a}{3}-\frac{b}{3}\\
0 & 0 & 0 & 3\,a-2\,b+c
\end{array}\right)$
\end{matlabsymbolicoutput}
\begin{matlabsymbolicoutput}
ans = 
    $\displaystyle \left(\begin{array}{cccc}
1 & 0 & \frac{7}{3} & 0\\
0 & 1 & -\frac{5}{3} & 0\\
0 & 0 & 0 & 1
\end{array}\right)$
\end{matlabsymbolicoutput}
\begin{par}
\begin{flushleft}
jalaba desde la matriz escalonada
\end{flushleft}
\end{par}

\begin{par}
\begin{flushleft}
correcto por rref
\end{flushleft}
\end{par}


\label{H_3D97F326}
\matlabheadingthree{Ej 5}

\begin{par}
\hfill \break
\end{par}

\begin{matlabsymbolicoutput}
A = 
    $\displaystyle \left(\begin{array}{cccc}
1 & -1 & 2 & 3\\
2 & -2 & 5 & 4\\
1 & 2 & -1 & -3\\
0 & 2 & 2 & 1
\end{array}\right)$
\end{matlabsymbolicoutput}
\begin{matlabsymbolicoutput}
A1 = 
    $\displaystyle \left(\begin{array}{cccc}
1 & -1 & 2 & 3\\
0 & 0 & 1 & -2\\
1 & 2 & -1 & -3\\
0 & 2 & 2 & 1
\end{array}\right)$
\end{matlabsymbolicoutput}
\begin{matlabsymbolicoutput}
A2 = 
    $\displaystyle \left(\begin{array}{cccc}
1 & -1 & 2 & 3\\
0 & 0 & 1 & -2\\
0 & 3 & -3 & -6\\
0 & 2 & 2 & 1
\end{array}\right)$
\end{matlabsymbolicoutput}
\begin{matlabsymbolicoutput}
A3 = 
    $\displaystyle \left(\begin{array}{cccc}
1 & -1 & 2 & 3\\
0 & 3 & -3 & -6\\
0 & 0 & 1 & -2\\
0 & 2 & 2 & 1
\end{array}\right)$
\end{matlabsymbolicoutput}
\begin{matlabsymbolicoutput}
A4 = 
    $\displaystyle \left(\begin{array}{cccc}
1 & -1 & 2 & 3\\
0 & 1 & -1 & -2\\
0 & 0 & 1 & -2\\
0 & 2 & 2 & 1
\end{array}\right)$
\end{matlabsymbolicoutput}
\begin{matlabsymbolicoutput}
A5 = 
    $\displaystyle \left(\begin{array}{cccc}
1 & -1 & 2 & 3\\
0 & 1 & -1 & -2\\
0 & 0 & 1 & -2\\
0 & 0 & 4 & 5
\end{array}\right)$
\end{matlabsymbolicoutput}
\begin{matlabsymbolicoutput}
A6 = 
    $\displaystyle \left(\begin{array}{cccc}
1 & -1 & 2 & 3\\
0 & 1 & -1 & -2\\
0 & 0 & 1 & -2\\
0 & 0 & 0 & 13
\end{array}\right)$
\end{matlabsymbolicoutput}
\begin{matlabsymbolicoutput}
A7 = 
    $\displaystyle \left(\begin{array}{cccc}
1 & 0 & 1 & 1\\
0 & 1 & -1 & -2\\
0 & 0 & 1 & -2\\
0 & 0 & 0 & 13
\end{array}\right)$
\end{matlabsymbolicoutput}
\begin{matlabsymbolicoutput}
A8 = 
    $\displaystyle \left(\begin{array}{cccc}
1 & 0 & 1 & 1\\
1 & 1 & 0 & -1\\
0 & 0 & 1 & -2\\
0 & 0 & 0 & 13
\end{array}\right)$
\end{matlabsymbolicoutput}
\begin{matlabsymbolicoutput}
A9 = 
    $\displaystyle \left(\begin{array}{cccc}
1 & 0 & 0 & 3\\
1 & 1 & 0 & -1\\
0 & 0 & 1 & -2\\
0 & 0 & 0 & 13
\end{array}\right)$
\end{matlabsymbolicoutput}
\begin{matlabsymbolicoutput}
ans = 
    $\displaystyle \left(\begin{array}{cccc}
1 & 0 & 0 & 0\\
0 & 1 & 0 & 0\\
0 & 0 & 1 & 0\\
0 & 0 & 0 & 1
\end{array}\right)$
\end{matlabsymbolicoutput}
\begin{par}
\begin{flushleft}
inconsistente
\end{flushleft}
\end{par}

\begin{par}
\begin{flushleft}
correcto por rref
\end{flushleft}
\end{par}


\label{H_14EFFA66}
\matlabheadingthree{Ej 6}

\begin{par}
\begin{flushleft}
a) falso -\textgreater{} tendria m renglones
\end{flushleft}
\end{par}

\begin{par}
\begin{flushleft}
b) falso -\textgreater{} uno inconsistente deberia tener una infinidad de soluciones
\end{flushleft}
\end{par}

\begin{par}
\begin{flushleft}
c) verdadero
\end{flushleft}
\end{par}

\begin{par}
\begin{flushleft}
d) verdadero
\end{flushleft}
\end{par}

\begin{par}
\begin{flushleft}
e) verdadero por transividad
\end{flushleft}
\end{par}


\label{H_E234252F}
\matlabheadingthree{Ej 7}

\begin{par}
\hfill \break
\end{par}

\begin{matlabsymbolicoutput}
A = 
    $\displaystyle \left(\begin{array}{cccc}
a & -b & c & 6\\
a & 2\,b & 4\,c & 0\\
a & 3\,b & 9\,c & 2
\end{array}\right)$
\end{matlabsymbolicoutput}
\begin{matlabsymbolicoutput}
A1 = 
    $\displaystyle \left(\begin{array}{cccc}
a & -b & c & 6\\
0 & 3\,b & 3\,c & -6\\
a & 3\,b & 9\,c & 2
\end{array}\right)$
\end{matlabsymbolicoutput}
\begin{matlabsymbolicoutput}
A2 = 
    $\displaystyle \left(\begin{array}{cccc}
a & -b & c & 6\\
0 & 3\,b & 3\,c & -6\\
0 & 4\,b & 8\,c & -4
\end{array}\right)$
\end{matlabsymbolicoutput}
\begin{matlabsymbolicoutput}
A3 = 
    $\displaystyle \left(\begin{array}{cccc}
a & -b & c & 6\\
0 & b & c & -2\\
0 & 4\,b & 8\,c & -4
\end{array}\right)$
\end{matlabsymbolicoutput}
\begin{matlabsymbolicoutput}
A4 = 
    $\displaystyle \left(\begin{array}{cccc}
a & -b & c & 6\\
0 & b & c & -2\\
0 & 0 & 4\,c & 4
\end{array}\right)$
\end{matlabsymbolicoutput}
\begin{matlabsymbolicoutput}
A5 = 
    $\displaystyle \left(\begin{array}{cccc}
a & -b & c & 6\\
0 & b & c & -2\\
0 & 0 & c & 1
\end{array}\right)$
\end{matlabsymbolicoutput}
\begin{matlabsymbolicoutput}
A6 = 
    $\displaystyle \left(\begin{array}{cccc}
a & -b & 0 & 5\\
0 & b & c & -2\\
0 & 0 & c & 1
\end{array}\right)$
\end{matlabsymbolicoutput}
\begin{matlabsymbolicoutput}
A7 = 
    $\displaystyle \left(\begin{array}{cccc}
a & -b & 0 & 5\\
0 & b & 0 & -3\\
0 & 0 & c & 1
\end{array}\right)$
\end{matlabsymbolicoutput}
\begin{matlabsymbolicoutput}
A8 = 
    $\displaystyle \left(\begin{array}{cccc}
a & 0 & 0 & 2\\
0 & b & 0 & -3\\
0 & 0 & c & 1
\end{array}\right)$
\end{matlabsymbolicoutput}
\begin{par}
\begin{flushleft}
la cuadratica seria 2-3x+x\textasciicircum{}2
\end{flushleft}
\end{par}


\label{H_2D6DEB1C}
\matlabheadingthree{Ej 8}

\begin{par}
\hfill \break
\end{par}

\begin{par}
\begin{flushleft}
\textbf{sin resolver}
\end{flushleft}
\end{par}


\label{H_8892C841}
\matlabheadingthree{Ej 9}

\begin{par}
\hfill \break
\end{par}

\begin{matlabsymbolicoutput}
A = 
    $\displaystyle \left(\begin{array}{ccc}
2 & 1 & 5\\
4 & -2 & t
\end{array}\right)$
\end{matlabsymbolicoutput}
\begin{par}
\begin{flushleft}
a)
\end{flushleft}
\end{par}

\begin{matlabsymbolicoutput}
A1 = 
    $\displaystyle \left(\begin{array}{ccc}
2 & 1 & 5\\
0 & -4 & t-10
\end{array}\right)$
\end{matlabsymbolicoutput}
\begin{matlabsymbolicoutput}
A2 = 
    $\displaystyle \left(\begin{array}{ccc}
2 & 0 & \frac{t}{4}+\frac{5}{2}\\
0 & -4 & t-10
\end{array}\right)$
\end{matlabsymbolicoutput}
\begin{matlabsymbolicoutput}
A3 = 
    $\displaystyle \left(\begin{array}{ccc}
1 & 0 & \frac{t}{8}+\frac{5}{4}\\
0 & -4 & t-10
\end{array}\right)$
\end{matlabsymbolicoutput}
\begin{matlabsymbolicoutput}
A4 = 
    $\displaystyle \left(\begin{array}{ccc}
1 & 0 & \frac{t}{8}+\frac{5}{4}\\
0 & 1 & \frac{5}{2}-\frac{t}{4}
\end{array}\right)$
\end{matlabsymbolicoutput}
\begin{par}
\begin{flushleft}
\textbf{definitivamente no se que pedo}
\end{flushleft}
\end{par}

\begin{par}
\begin{flushleft}
b) tampoco en esta
\end{flushleft}
\end{par}


\label{H_CF1772B1}
\matlabheadingthree{Ej 10}

\begin{par}
\hfill \break
\end{par}

\begin{matlabsymbolicoutput}
A = 
    $\displaystyle \left(\begin{array}{cccc}
1 & 3 & 1 & a\\
-1 & -2 & 1 & b\\
3 & 7 & -1 & c
\end{array}\right)$
\end{matlabsymbolicoutput}
\begin{matlabsymbolicoutput}
A1 = 
    $\displaystyle \left(\begin{array}{cccc}
1 & 3 & 1 & a\\
0 & 1 & 2 & a+b\\
3 & 7 & -1 & c
\end{array}\right)$
\end{matlabsymbolicoutput}
\begin{matlabsymbolicoutput}
A2 = 
    $\displaystyle \left(\begin{array}{cccc}
1 & 3 & 1 & a\\
0 & 1 & 2 & a+b\\
0 & -2 & -4 & c-3\,a
\end{array}\right)$
\end{matlabsymbolicoutput}
\begin{matlabsymbolicoutput}
A3 = 
    $\displaystyle \left(\begin{array}{cccc}
1 & 0 & -5 & -2\,a-3\,b\\
0 & 1 & 2 & a+b\\
0 & -2 & -4 & c-3\,a
\end{array}\right)$
\end{matlabsymbolicoutput}
\begin{matlabsymbolicoutput}
A4 = 
    $\displaystyle \left(\begin{array}{cccc}
1 & 0 & -5 & -2\,a-3\,b\\
0 & 1 & 2 & a+b\\
0 & 0 & 0 & 2\,b-a+c
\end{array}\right)$
\end{matlabsymbolicoutput}
\begin{par}
\begin{flushleft}
2*b-a+c=0 para que sea consistente
\end{flushleft}
\end{par}


\label{H_0618CE1C}
\matlabheadingthree{Ej 11}

\begin{par}
\begin{flushleft}
a) 
\end{flushleft}
\end{par}

\begin{par}
\begin{flushleft}
 3 variables y solo 2 ecuaciones sin solucion
\end{flushleft}
\end{par}

\begin{matlabsymbolicoutput}
A = 
    $\displaystyle \left(\begin{array}{cccc}
1 & 1 & 1 & 1\\
0 & 0 & 0 & 1
\end{array}\right)$
\end{matlabsymbolicoutput}
\begin{par}
\begin{flushleft}
b)    
\end{flushleft}
\end{par}

\begin{par}
\begin{flushleft}
2 variables 3 ecuaciones \textbf{(no se si tiene solucion unica)}
\end{flushleft}
\end{par}

\begin{matlabsymbolicoutput}
A = 
    $\displaystyle \left(\begin{array}{ccc}
1 & 1 & 1\\
1 & 1 & 1\\
1 & 1 & 1
\end{array}\right)$
\end{matlabsymbolicoutput}

\label{H_1D6D04CA}
\matlabheadingthree{Ej 12}

\begin{par}
\begin{flushleft}
(3,4,-2) (como se si es la unica)
\end{flushleft}
\end{par}

\begin{matlabsymbolicoutput}
A = 
    $\displaystyle \left(\begin{array}{cccc}
5 & -1 & 2 & 7\\
-2 & 6 & 9 & 0\\
-7 & 5 & -3 & -7
\end{array}\right)$
\end{matlabsymbolicoutput}
\begin{matlabsymbolicoutput}
A1 = 
    $\displaystyle \left(\begin{array}{cccc}
1 & -\frac{1}{5} & \frac{2}{5} & \frac{7}{5}\\
-2 & 6 & 9 & 0\\
-7 & 5 & -3 & -7
\end{array}\right)$
\end{matlabsymbolicoutput}
\begin{matlabsymbolicoutput}
A2 = 
    $\displaystyle \left(\begin{array}{cccc}
1 & -\frac{1}{5} & \frac{2}{5} & \frac{7}{5}\\
0 & \frac{28}{5} & \frac{49}{5} & \frac{14}{5}\\
-7 & 5 & -3 & -7
\end{array}\right)$
\end{matlabsymbolicoutput}
\begin{matlabsymbolicoutput}
A3 = 
    $\displaystyle \left(\begin{array}{cccc}
1 & -\frac{1}{5} & \frac{2}{5} & \frac{7}{5}\\
0 & \frac{28}{5} & \frac{49}{5} & \frac{14}{5}\\
0 & \frac{18}{5} & -\frac{1}{5} & \frac{14}{5}
\end{array}\right)$
\end{matlabsymbolicoutput}
\begin{matlabsymbolicoutput}
A4 = 
    $\displaystyle \left(\begin{array}{cccc}
1 & -\frac{1}{5} & \frac{2}{5} & \frac{7}{5}\\
0 & 1 & \frac{7}{4} & \frac{1}{2}\\
0 & \frac{18}{5} & -\frac{1}{5} & \frac{14}{5}
\end{array}\right)$
\end{matlabsymbolicoutput}
\begin{matlabsymbolicoutput}
A5 = 
    $\displaystyle \left(\begin{array}{cccc}
1 & 0 & \frac{3}{4} & \frac{3}{2}\\
0 & 1 & \frac{7}{4} & \frac{1}{2}\\
0 & \frac{18}{5} & -\frac{1}{5} & \frac{14}{5}
\end{array}\right)$
\end{matlabsymbolicoutput}
\begin{matlabsymbolicoutput}
A6 = 
    $\displaystyle \left(\begin{array}{cccc}
1 & 0 & \frac{3}{4} & \frac{3}{2}\\
0 & 1 & \frac{7}{4} & \frac{1}{2}\\
0 & 0 & -\frac{13}{2} & 1
\end{array}\right)$
\end{matlabsymbolicoutput}
\begin{matlabsymbolicoutput}
A7 = 
    $\displaystyle \left(\begin{array}{cccc}
1 & 0 & \frac{3}{4} & \frac{3}{2}\\
0 & 1 & \frac{7}{4} & \frac{1}{2}\\
0 & 0 & 1 & -\frac{2}{13}
\end{array}\right)$
\end{matlabsymbolicoutput}
\begin{matlabsymbolicoutput}
A8 = 
    $\displaystyle \left(\begin{array}{cccc}
1 & 0 & \frac{3}{4} & \frac{3}{2}\\
0 & 1 & 0 & \frac{10}{13}\\
0 & 0 & 1 & -\frac{2}{13}
\end{array}\right)$
\end{matlabsymbolicoutput}
\begin{matlabsymbolicoutput}
A9 = 
    $\displaystyle \left(\begin{array}{cccc}
1 & 0 & 0 & \frac{21}{13}\\
0 & 1 & 0 & \frac{10}{13}\\
0 & 0 & 1 & -\frac{2}{13}
\end{array}\right)$
\end{matlabsymbolicoutput}
\begin{par}
\begin{flushleft}
\textbf{not a clue how to procede}
\end{flushleft}
\end{par}


\label{H_A94473BA}
\matlabheadingthree{Ej 13}

\begin{par}
\hfill \break
\end{par}

\begin{matlabsymbolicoutput}
A = 
    $\displaystyle \left(\begin{array}{ccc}
2 & -1 & h\\
-6 & -3 & k
\end{array}\right)$
\end{matlabsymbolicoutput}
\begin{matlabsymbolicoutput}
A1 = 
    $\displaystyle \left(\begin{array}{ccc}
2 & -1 & h\\
0 & -6 & 3\,h+k
\end{array}\right)$
\end{matlabsymbolicoutput}
\begin{matlabsymbolicoutput}
A2 = 
    $\displaystyle \left(\begin{array}{ccc}
1 & -\frac{1}{2} & \frac{h}{2}\\
0 & -6 & 3\,h+k
\end{array}\right)$
\end{matlabsymbolicoutput}
\begin{matlabsymbolicoutput}
A3 = 
    $\displaystyle \left(\begin{array}{ccc}
1 & -\frac{1}{2} & \frac{h}{2}\\
0 & 1 & -\frac{h}{2}-\frac{k}{6}
\end{array}\right)$
\end{matlabsymbolicoutput}
\begin{matlabsymbolicoutput}
A4 = 
    $\displaystyle \left(\begin{array}{ccc}
1 & 0 & \frac{h}{4}-\frac{k}{12}\\
0 & 1 & -\frac{h}{2}-\frac{k}{6}
\end{array}\right)$
\end{matlabsymbolicoutput}
\begin{par}
\begin{flushleft}
\textbf{para todos, no?}
\end{flushleft}
\end{par}


\label{H_2C27C561}
\matlabheadingthree{Ej 14}

\begin{par}
\begin{flushleft}
a)
\end{flushleft}
\end{par}

\begin{matlabsymbolicoutput}
A = 
    $\displaystyle \left(\begin{array}{cccc}
0 & 1 & 4 & -5\\
1 & 3 & 5 & -2\\
3 & 7 & 7 & 6
\end{array}\right)$
\end{matlabsymbolicoutput}
\begin{matlabsymbolicoutput}
A1 = 
    $\displaystyle \left(\begin{array}{cccc}
1 & 3 & 5 & -2\\
0 & 1 & 4 & -5\\
3 & 7 & 7 & 6
\end{array}\right)$
\end{matlabsymbolicoutput}
\begin{matlabsymbolicoutput}
A2 = 
    $\displaystyle \left(\begin{array}{cccc}
1 & 3 & 5 & -2\\
0 & 1 & 4 & -5\\
0 & -2 & -8 & 12
\end{array}\right)$
\end{matlabsymbolicoutput}
\begin{matlabsymbolicoutput}
A3 = 
    $\displaystyle \left(\begin{array}{cccc}
1 & 3 & 5 & -2\\
0 & 1 & 4 & -5\\
0 & 0 & 0 & 2
\end{array}\right)$
\end{matlabsymbolicoutput}
\begin{matlabsymbolicoutput}
A4 = 
    $\displaystyle \left(\begin{array}{cccc}
1 & 0 & -7 & 13\\
0 & 1 & 4 & -5\\
0 & 0 & 0 & 2
\end{array}\right)$
\end{matlabsymbolicoutput}
\begin{matlabsymbolicoutput}
ans = 
    $\displaystyle \left(\begin{array}{cccc}
1 & 0 & -7 & 0\\
0 & 1 & 4 & 0\\
0 & 0 & 0 & 1
\end{array}\right)$
\end{matlabsymbolicoutput}
\begin{par}
\begin{flushleft}
no tiene solucion
\end{flushleft}
\end{par}

\begin{par}
\begin{flushleft}
com rref
\end{flushleft}
\end{par}


\begin{par}
\begin{flushleft}
b)
\end{flushleft}
\end{par}

\begin{matlabsymbolicoutput}
A = 
    $\displaystyle \left(\begin{array}{cccc}
1 & -3 & 4 & -4\\
3 & -7 & 7 & -8\\
-4 & 6 & -1 & 7
\end{array}\right)$
\end{matlabsymbolicoutput}
\begin{matlabsymbolicoutput}
A1 = 
    $\displaystyle \left(\begin{array}{cccc}
1 & -3 & 4 & -4\\
0 & 2 & -5 & 4\\
-4 & 6 & -1 & 7
\end{array}\right)$
\end{matlabsymbolicoutput}
\begin{matlabsymbolicoutput}
A2 = 
    $\displaystyle \left(\begin{array}{cccc}
1 & -3 & 4 & -4\\
0 & 2 & -5 & 4\\
0 & -6 & 15 & -9
\end{array}\right)$
\end{matlabsymbolicoutput}
\begin{matlabsymbolicoutput}
A3 = 
    $\displaystyle \left(\begin{array}{cccc}
1 & -3 & 4 & -4\\
0 & 1 & -\frac{5}{2} & 2\\
0 & -6 & 15 & -9
\end{array}\right)$
\end{matlabsymbolicoutput}
\begin{matlabsymbolicoutput}
A4 = 
    $\displaystyle \left(\begin{array}{cccc}
1 & 0 & -\frac{7}{2} & 2\\
0 & 1 & -\frac{5}{2} & 2\\
0 & -6 & 15 & -9
\end{array}\right)$
\end{matlabsymbolicoutput}
\begin{matlabsymbolicoutput}
A5 = 
    $\displaystyle \left(\begin{array}{cccc}
1 & 0 & -\frac{7}{2} & 2\\
0 & 1 & -\frac{5}{2} & 2\\
0 & 0 & 0 & 3
\end{array}\right)$
\end{matlabsymbolicoutput}
\begin{matlabsymbolicoutput}
ans = 
    $\displaystyle \left(\begin{array}{cccc}
1 & 0 & -\frac{7}{2} & 0\\
0 & 1 & -\frac{5}{2} & 0\\
0 & 0 & 0 & 1
\end{array}\right)$
\end{matlabsymbolicoutput}
\begin{par}
\begin{flushleft}
no tiene solucion
\end{flushleft}
\end{par}

\begin{par}
\begin{flushleft}
com rref
\end{flushleft}
\end{par}


\begin{par}
\begin{flushleft}
c)
\end{flushleft}
\end{par}

\begin{matlabsymbolicoutput}
A = 
    $\displaystyle \left(\begin{array}{cccc}
1 & 0 & -3 & 8\\
2 & 2 & 9 & 7\\
0 & 1 & 5 & -2
\end{array}\right)$
\end{matlabsymbolicoutput}
\begin{matlabsymbolicoutput}
A1 = 
    $\displaystyle \left(\begin{array}{cccc}
1 & 0 & -3 & 8\\
0 & 1 & 5 & -2\\
2 & 2 & 9 & 7
\end{array}\right)$
\end{matlabsymbolicoutput}
\begin{matlabsymbolicoutput}
A2 = 
    $\displaystyle \left(\begin{array}{cccc}
1 & 0 & -3 & 8\\
0 & 1 & 5 & -2\\
0 & 2 & 15 & -9
\end{array}\right)$
\end{matlabsymbolicoutput}
\begin{matlabsymbolicoutput}
A3 = 
    $\displaystyle \left(\begin{array}{cccc}
1 & 0 & -3 & 8\\
0 & 1 & 5 & -2\\
0 & 0 & 5 & -5
\end{array}\right)$
\end{matlabsymbolicoutput}
\begin{matlabsymbolicoutput}
A4 = 
    $\displaystyle \left(\begin{array}{cccc}
1 & 0 & -3 & 8\\
0 & 1 & 5 & -2\\
0 & 0 & 1 & -1
\end{array}\right)$
\end{matlabsymbolicoutput}
\begin{matlabsymbolicoutput}
A5 = 
    $\displaystyle \left(\begin{array}{cccc}
1 & 0 & -3 & 8\\
0 & 1 & 0 & 3\\
0 & 0 & 1 & -1
\end{array}\right)$
\end{matlabsymbolicoutput}
\begin{matlabsymbolicoutput}
A6 = 
    $\displaystyle \left(\begin{array}{cccc}
1 & 0 & 0 & 5\\
0 & 1 & 0 & 3\\
0 & 0 & 1 & -1
\end{array}\right)$
\end{matlabsymbolicoutput}
\begin{matlabsymbolicoutput}
ans = 
    $\displaystyle \left(\begin{array}{cccc}
1 & 0 & 0 & 5\\
0 & 1 & 0 & 3\\
0 & 0 & 1 & -1
\end{array}\right)$
\end{matlabsymbolicoutput}
\begin{par}
\begin{flushleft}
si habemus respuesta
\end{flushleft}
\end{par}

\begin{par}
\begin{flushleft}
com rref
\end{flushleft}
\end{par}


\label{H_02AAA586}
\matlabheadingthree{Ej 15}

\begin{par}
\hfill \break
\end{par}

\begin{matlabsymbolicoutput}
A = 
    $\displaystyle \left(\begin{array}{ccc}
1 & -4 & 1\\
2 & -1 & -3\\
-1 & -3 & 4
\end{array}\right)$
\end{matlabsymbolicoutput}
\begin{matlabsymbolicoutput}
A1 = 
    $\displaystyle \left(\begin{array}{ccc}
1 & -4 & 1\\
2 & -1 & -3\\
0 & -7 & 5
\end{array}\right)$
\end{matlabsymbolicoutput}
\begin{matlabsymbolicoutput}
A2 = 
    $\displaystyle \left(\begin{array}{ccc}
1 & -4 & 1\\
0 & 7 & -5\\
0 & -7 & 5
\end{array}\right)$
\end{matlabsymbolicoutput}
\begin{matlabsymbolicoutput}
A3 = 
    $\displaystyle \left(\begin{array}{ccc}
1 & -4 & 1\\
0 & 7 & -5\\
0 & 0 & 0
\end{array}\right)$
\end{matlabsymbolicoutput}
\begin{matlabsymbolicoutput}
A4 = 
    $\displaystyle \left(\begin{array}{ccc}
1 & -4 & 1\\
0 & 1 & -\frac{5}{7}\\
0 & 0 & 0
\end{array}\right)$
\end{matlabsymbolicoutput}
\begin{matlabsymbolicoutput}
A5 = 
    $\displaystyle \left(\begin{array}{ccc}
1 & 0 & -\frac{13}{7}\\
0 & 1 & -\frac{5}{7}\\
0 & 0 & 0
\end{array}\right)$
\end{matlabsymbolicoutput}
\begin{par}
\begin{flushleft}
\textbf{creo que si?? pero cual es el punto en x\_3}
\end{flushleft}
\end{par}


\label{H_56631144}
\matlabheadingthree{Ej 16}

\begin{par}
\hfill \break
\end{par}

\begin{matlabsymbolicoutput}
A = 
    $\displaystyle \left(\begin{array}{cccc}
1 & 2 & 1 & 4\\
0 & 1 & -1 & 1\\
1 & 3 & 0 & 0
\end{array}\right)$
\end{matlabsymbolicoutput}
\begin{matlabsymbolicoutput}
A1 = 
    $\displaystyle \left(\begin{array}{cccc}
1 & 2 & 1 & 4\\
0 & 1 & -1 & 1\\
0 & 1 & -1 & -4
\end{array}\right)$
\end{matlabsymbolicoutput}
\begin{matlabsymbolicoutput}
A2 = 
    $\displaystyle \left(\begin{array}{cccc}
1 & 0 & 3 & 2\\
0 & 1 & -1 & 1\\
0 & 1 & -1 & -4
\end{array}\right)$
\end{matlabsymbolicoutput}
\begin{matlabsymbolicoutput}
A3 = 
    $\displaystyle \left(\begin{array}{cccc}
1 & 0 & 3 & 2\\
0 & 1 & -1 & 1\\
0 & 0 & 0 & -5
\end{array}\right)$
\end{matlabsymbolicoutput}
\begin{par}
\begin{flushleft}
el sistema es inconsistente entonces supongo que no?
\end{flushleft}
\end{par}


\label{H_418257EC}
\matlabheadingthree{Ej 17}

\begin{par}
\hfill \break
\end{par}

\begin{par}
\begin{flushleft}
a)
\end{flushleft}
\end{par}

\begin{matlabsymbolicoutput}
A = 
    $\displaystyle \left(\begin{array}{ccc}
1 & h & -3\\
-2 & 4 & 6
\end{array}\right)$
\end{matlabsymbolicoutput}
\begin{matlabsymbolicoutput}
A1 = 
    $\displaystyle \left(\begin{array}{ccc}
1 & h & -3\\
1 & -2 & -3
\end{array}\right)$
\end{matlabsymbolicoutput}
\begin{matlabsymbolicoutput}
A2 = 
    $\displaystyle \left(\begin{array}{ccc}
1 & h & -3\\
0 & -h-2 & 0
\end{array}\right)$
\end{matlabsymbolicoutput}
\begin{par}
\begin{flushleft}
ni siquiera puede inconsistentar el sistema 
\end{flushleft}
\end{par}

\begin{par}
\begin{flushleft}
b)
\end{flushleft}
\end{par}

\begin{par}
\begin{flushleft}
mismo que a)
\end{flushleft}
\end{par}

\begin{par}
\begin{flushleft}
c)
\end{flushleft}
\end{par}

\begin{par}
\begin{flushleft}
mismo que a)
\end{flushleft}
\end{par}

\begin{par}
\begin{flushleft}
d)
\end{flushleft}
\end{par}

\begin{par}
\begin{flushleft}
mismo que a)
\end{flushleft}
\end{par}

\begin{par}
\begin{flushleft}
\textbf{seguro I am missing something que si pueda inconsistentar el sistema}
\end{flushleft}
\end{par}


\label{H_6719AD61}
\matlabheadingthree{Ej 18}

\label{H_466FB2A3}
\begin{par}
\begin{flushleft}
ni hay sistema presentado a continuacion
\end{flushleft}
\end{par}


\label{H_07420CED}
\matlabheadingthree{Ej 19}

\label{H_AD9419F5}
\begin{par}
\begin{flushleft}
tampoco hay sistema de ecuaciones a continuacion
\end{flushleft}
\end{par}


\label{H_3BAF66ED}
\matlabheadingthree{Ej 20}

\begin{par}
\hfill \break
\end{par}

\begin{matlabsymbolicoutput}
A = 
    $\displaystyle \left(\begin{array}{ccc}
a & b & 0\\
c & d & 0
\end{array}\right)$
\end{matlabsymbolicoutput}
\begin{par}
\begin{flushleft}
\textbf{not a clue}
\end{flushleft}
\end{par}


\label{H_36826855}
\matlabheadingthree{Ej 21}

\begin{par}
\hfill \break
\end{par}

\begin{matlabsymbolicoutput}
A = 
    $\displaystyle \left(\begin{array}{cccc}
a & b & c & -5\\
a & -b & c & 1\\
4\,a & 2\,b & c & 7
\end{array}\right)$
\end{matlabsymbolicoutput}
\begin{matlabsymbolicoutput}
A1 = 
    $\displaystyle \left(\begin{array}{cccc}
a & b & c & -5\\
0 & -2\,b & 0 & 6\\
4\,a & 2\,b & c & 7
\end{array}\right)$
\end{matlabsymbolicoutput}
\begin{matlabsymbolicoutput}
A2 = 
    $\displaystyle \left(\begin{array}{cccc}
a & b & c & -5\\
0 & -2\,b & 0 & 6\\
0 & -2\,b & -3\,c & 27
\end{array}\right)$
\end{matlabsymbolicoutput}
\begin{matlabsymbolicoutput}
A3 = 
    $\displaystyle \left(\begin{array}{cccc}
a & b & c & -5\\
0 & b & 0 & -3\\
0 & -2\,b & -3\,c & 27
\end{array}\right)$
\end{matlabsymbolicoutput}
\begin{matlabsymbolicoutput}
A4 = 
    $\displaystyle \left(\begin{array}{cccc}
a & 0 & c & -2\\
0 & b & 0 & -3\\
0 & -2\,b & -3\,c & 27
\end{array}\right)$
\end{matlabsymbolicoutput}
\begin{matlabsymbolicoutput}
A5 = 
    $\displaystyle \left(\begin{array}{cccc}
a & 0 & c & -2\\
0 & b & 0 & -3\\
0 & 0 & -3\,c & 21
\end{array}\right)$
\end{matlabsymbolicoutput}
\begin{matlabsymbolicoutput}
A6 = 
    $\displaystyle \left(\begin{array}{cccc}
a & 0 & c & -2\\
0 & b & 0 & -3\\
0 & 0 & c & -7
\end{array}\right)$
\end{matlabsymbolicoutput}
\begin{matlabsymbolicoutput}
A7 = 
    $\displaystyle \left(\begin{array}{cccc}
a & 0 & 0 & 5\\
0 & b & 0 & -3\\
0 & 0 & c & -7
\end{array}\right)$
\end{matlabsymbolicoutput}
\begin{par}
\begin{flushleft}
la parabola seria 
\end{flushleft}
\end{par}

\begin{par}
\begin{flushleft}
p(x) = 5x\textasciicircum{}2 - 3x -7
\end{flushleft}
\end{par}


\label{H_1579D558}
\matlabheadingthree{Ej 22}

\begin{par}
\begin{flushleft}
sin azufre:     1 ton = 5 m + 4 r
\end{flushleft}
\end{par}

\begin{par}
\begin{flushleft}
con azufre:    1 ton = 4 m + 2 r
\end{flushleft}
\end{par}

\begin{par}
\begin{flushleft}
planta M = 3 horas
\end{flushleft}
\end{par}

\begin{par}
\begin{flushleft}
planta R = 2 horas
\end{flushleft}
\end{par}

\begin{par}
\begin{flushleft}
\textbf{la neta no se como taclear este problema}
\end{flushleft}
\end{par}

\label{H_203EE86F}
\matlabheading{Tarea 2}


\label{H_9CA92D12}
\matlabheadingthree{Ej 1}

\begin{par}
\begin{flushleft}
\includegraphics[width=\maxwidth{46.06121424987456em}]{image_2}
\end{flushleft}
\end{par}

\begin{matlabsymbolicoutput}
ans = 
    $\displaystyle \left(\begin{array}{cc}
5 & 4\\
-1 & 7\\
9 & -3
\end{array}\right)$
\end{matlabsymbolicoutput}
\begin{matlabsymbolicoutput}
ans = 
    $\displaystyle \left(\begin{array}{cc}
-3 & 0\\
4 & 2\\
5 & -7
\end{array}\right)$
\end{matlabsymbolicoutput}
\begin{par}
\begin{flushleft}
a) \includegraphics[width=\maxwidth{5.3186151530356245em}]{image_3}
\end{flushleft}
\end{par}

\begin{matlabsymbolicoutput}
ans = 
    $\displaystyle \left(\begin{array}{cc}
2 & 4\\
3 & 9\\
14 & -10
\end{array}\right)$
\end{matlabsymbolicoutput}
\begin{par}
\begin{flushleft}
b)\includegraphics[width=\maxwidth{6.623181133968891em}]{image_4}
\end{flushleft}
\end{par}

\begin{matlabsymbolicoutput}
ans = 
    $\displaystyle \left(\begin{array}{cc}
7 & 8\\
2 & 16\\
23 & -13
\end{array}\right)$
\end{matlabsymbolicoutput}
\begin{par}
\begin{flushleft}
c) $A^T$
\end{flushleft}
\end{par}

\begin{matlabsymbolicoutput}
ans = 
    $\displaystyle \left(\begin{array}{ccc}
5 & -1 & 9\\
4 & 7 & -3
\end{array}\right)$
\end{matlabsymbolicoutput}

\label{H_C3154DAF}
\matlabheadingthree{Ej 2}

\begin{par}
\hfill \break
\end{par}

\begin{par}
\begin{flushleft}
a)    
\end{flushleft}
\end{par}

\begin{matlabsymbolicoutput}
ans = 
    $\displaystyle \left(\begin{array}{ccc}
9 & 8 & 13\\
8 & -6 & 4\\
-5 & 4 & 8
\end{array}\right)$
\end{matlabsymbolicoutput}
\begin{par}
\begin{flushleft}
b)
\end{flushleft}
\end{par}

\begin{matlabsymbolicoutput}
ans = 
    $\displaystyle \left(\begin{array}{c}
21\\
6\\
1
\end{array}\right)$
\end{matlabsymbolicoutput}
\begin{par}
\begin{flushleft}
c)
\end{flushleft}
\end{par}

\begin{matlabsymbolicoutput}
ans = 
    $\displaystyle \left(\begin{array}{c}
-9\\
-2\\
1
\end{array}\right)$
\end{matlabsymbolicoutput}
\begin{par}
\begin{flushleft}
d)
\end{flushleft}
\end{par}

\begin{matlabsymbolicoutput}
ans = 
    $\displaystyle \left(\begin{array}{ccc}
0 & 6 & -4\\
-1 & -8 & 5\\
4 & 2 & 9
\end{array}\right)$
\end{matlabsymbolicoutput}

\matlabheadingthree{Ej 3}

\begin{par}
\hfill \break
\end{par}

\begin{par}
\begin{flushleft}
a)
\end{flushleft}
\end{par}

\begin{matlaboutput}
ans = 
       0              1       
      -2              5       

\end{matlaboutput}
\begin{par}
\begin{flushleft}
b)
\end{flushleft}
\end{par}

\begin{matlaboutput}
ans = 
      -1              0              3       
       5              7              2       

\end{matlaboutput}
\begin{par}
\begin{flushleft}
c)
\end{flushleft}
\end{par}

\begin{matlaboutput}
ans = 
       1              0       
       0              1       

\end{matlaboutput}
\begin{par}
\begin{flushleft}
d)
\end{flushleft}
\end{par}

\begin{matlaboutput}
ans = 
       5              7              2       
      27             35              4       

\end{matlaboutput}
\begin{par}
\begin{flushleft}
e)
\end{flushleft}
\end{par}

\begin{matlaboutput}
ans = 
       0             -2       
       1              5       

\end{matlaboutput}

\matlabheadingthree{Ej 4}

\begin{par}
\hfill \break
\end{par}

\begin{par}
\begin{flushleft}
a)
\end{flushleft}
\end{par}

\begin{matlaboutput}
ans = 
     -12              2       
      -9            -24       
     -20            -24       

\end{matlaboutput}
\begin{par}
\begin{flushleft}
b) checa este pedo as well
\end{flushleft}
\end{par}

\begin{par}
\begin{flushleft}
c) checa este pedo
\end{flushleft}
\end{par}

\begin{par}
\begin{flushleft}
d)
\end{flushleft}
\end{par}

\begin{matlaboutput}
ans = 
     -12             -9            -20       
       2            -24            -24       

\end{matlaboutput}

\label{H_9E2F78A5}
\matlabheadingthree{Ej 5}

\begin{par}
\hfill \break
\end{par}

\begin{matlabsymbolicoutput}
C = 
    $\displaystyle \left(\begin{array}{ccc}
15 & 16 & -24\\
28 & 49 & 24\\
-1 & 0 & -15
\end{array}\right)$
\end{matlabsymbolicoutput}
\begin{matlabsymbolicoutput}
D = 
    $\displaystyle \left(\begin{array}{ccc}
1 & 9 & 17\\
16 & 57 & -8\\
8 & 6 & -9
\end{array}\right)$
\end{matlabsymbolicoutput}
\begin{par}
\begin{flushleft}
a)
\end{flushleft}
\end{par}

\begin{matlabsymbolicoutput}
ans = 
    $\displaystyle -1$
\end{matlabsymbolicoutput}
\begin{par}
\begin{flushleft}
b)
\end{flushleft}
\end{par}

\begin{matlabsymbolicoutput}
ans = 
    $\displaystyle 9$
\end{matlabsymbolicoutput}

\label{H_A7751211}
\matlabheadingthree{Ej 6}

\begin{par}
\hfill \break
\end{par}

\begin{matlabsymbolicoutput}
A = 
    $\displaystyle \left(\begin{array}{ccc}
a_{1,1}  & a_{1,2}  & a_{1,3} \\
a_{2,1}  & a_{2,2}  & a_{2,3} \\
0 & 0 & 0
\end{array}\right)$
\end{matlabsymbolicoutput}
\begin{matlabsymbolicoutput}
B = 
    $\displaystyle \left(\begin{array}{ccc}
b_{1,1}  & b_{1,2}  & b_{1,3} \\
b_{2,1}  & b_{2,2}  & b_{2,3} \\
b_{3,1}  & b_{3,2}  & b_{3,3} 
\end{array}\right)$
\end{matlabsymbolicoutput}
\begin{matlabsymbolicoutput}
C = 
    $\displaystyle \left(\begin{array}{ccc}
a_{1,1} \,b_{1,1} +a_{1,2} \,b_{2,1} +a_{1,3} \,b_{3,1}  & a_{1,1} \,b_{1,2} +a_{1,2} \,b_{2,2} +a_{1,3} \,b_{3,2}  & a_{1,1} \,b_{1,3} +a_{1,2} \,b_{2,3} +a_{1,3} \,b_{3,3} \\
a_{2,1} \,b_{1,1} +a_{2,2} \,b_{2,1} +a_{2,3} \,b_{3,1}  & a_{2,1} \,b_{1,2} +a_{2,2} \,b_{2,2} +a_{2,3} \,b_{3,2}  & a_{2,1} \,b_{1,3} +a_{2,2} \,b_{2,3} +a_{2,3} \,b_{3,3} \\
0 & 0 & 0
\end{array}\right)$
\end{matlabsymbolicoutput}
\begin{matlabsymbolicoutput}
C_31 = 
    $\displaystyle 0$
\end{matlabsymbolicoutput}
\begin{matlabsymbolicoutput}
C_32 = 
    $\displaystyle 0$
\end{matlabsymbolicoutput}
\begin{matlabsymbolicoutput}
C_33 = 
    $\displaystyle 0$
\end{matlabsymbolicoutput}

\vspace{1em}

\label{H_3022F90C}
\matlabheadingthree{Ej 7}

\begin{par}
\begin{flushleft}
Sea A una matriz de mxn tal que A = -A: Prueba que A = 0
\end{flushleft}
\end{par}

\begin{matlabsymbolicoutput}
A = 
    $\displaystyle \left(\begin{array}{ccc}
a_{1,1}  & a_{1,2}  & a_{1,3} \\
a_{2,1}  & a_{2,2}  & a_{2,3} \\
a_{3,1}  & a_{3,2}  & a_{3,3} 
\end{array}\right)$
\end{matlabsymbolicoutput}
\begin{matlabsymbolicoutput}
ans = 
    $\displaystyle \left(\begin{array}{ccc}
-a_{1,1}  & -a_{1,2}  & -a_{1,3} \\
-a_{2,1}  & -a_{2,2}  & -a_{2,3} \\
-a_{3,1}  & -a_{3,2}  & -a_{3,3} 
\end{array}\right)$
\end{matlabsymbolicoutput}
\begin{par}
\begin{flushleft}
Sabemos que A=-A si son iguales termino a termino, esto quiere decir que 
\end{flushleft}
\end{par}

\begin{par}
\begin{flushleft}
a11=-a11 ... a33=-a33
\end{flushleft}
\end{par}

\begin{par}
\begin{flushleft}
esto solo se puede dar si todos los valores de A son iguales a 0
\end{flushleft}
\end{par}

\begin{par}
\begin{flushleft}
la matriz que tiene todas sus elementos iguales a 0 a es la matriz 0
\end{flushleft}
\end{par}

\begin{par}
\begin{flushleft}
entonces A = 0
\end{flushleft}
\end{par}


\matlabheadingthree{Ej 8}

\begin{par}
\hfill \break
\end{par}

\begin{par}
\begin{flushleft}
a) 
\end{flushleft}
\end{par}

\begin{matlabsymbolicoutput}
A = 
    $\displaystyle \left(\begin{array}{cc}
a & b\\
c & d
\end{array}\right)$
\end{matlabsymbolicoutput}
\begin{matlabsymbolicoutput}
B = 
    $\displaystyle \left(\begin{array}{cc}
c-3\,d & -d\\
2\,a+d & a+b
\end{array}\right)$
\end{matlabsymbolicoutput}
\begin{matlabsymbolicoutput}
C = 
    $\displaystyle \left(\begin{array}{cc}
a & c-3\,d\\
b & -d\\
c & 2\,a+d\\
d & a+b
\end{array}\right)$
\end{matlabsymbolicoutput}
\begin{matlabsymbolicoutput}
C = 
    $\displaystyle \left(\begin{array}{cccc}
1 & 0 & 1 & -3\\
0 & 1 & 0 & -1\\
2 & 0 & 1 & 1\\
1 & 1 & 0 & 1
\end{array}\right)$
\end{matlabsymbolicoutput}
\begin{matlabsymbolicoutput}
C1 = 
    $\displaystyle \left(\begin{array}{cccc}
1 & 0 & 1 & -3\\
0 & 1 & 0 & -1\\
0 & 0 & -1 & 7\\
1 & 1 & 0 & 1
\end{array}\right)$
\end{matlabsymbolicoutput}
\begin{matlabsymbolicoutput}
C2 = 
    $\displaystyle \left(\begin{array}{cccc}
1 & 0 & 1 & -3\\
0 & 1 & 0 & -1\\
0 & 0 & -1 & 7\\
0 & 1 & -1 & 4
\end{array}\right)$
\end{matlabsymbolicoutput}
\begin{matlabsymbolicoutput}
C3 = 
    $\displaystyle \left(\begin{array}{cccc}
1 & 0 & 1 & -3\\
0 & 1 & 0 & -1\\
0 & 0 & -1 & 7\\
0 & 0 & -1 & 5
\end{array}\right)$
\end{matlabsymbolicoutput}
\begin{matlabsymbolicoutput}
C4 = 
    $\displaystyle \left(\begin{array}{cccc}
1 & 0 & 1 & -3\\
0 & 1 & 0 & -1\\
0 & 0 & -1 & 7\\
0 & 0 & 0 & -2
\end{array}\right)$
\end{matlabsymbolicoutput}
\begin{matlabsymbolicoutput}
C5 = 
    $\displaystyle \left(\begin{array}{cccc}
1 & 0 & 0 & 4\\
0 & 1 & 0 & -1\\
0 & 0 & -1 & 7\\
0 & 0 & 0 & -2
\end{array}\right)$
\end{matlabsymbolicoutput}
\begin{matlabsymbolicoutput}
C6 = 
    $\displaystyle \left(\begin{array}{cccc}
1 & 0 & 0 & 4\\
0 & 1 & 0 & -1\\
0 & 0 & 1 & -7\\
0 & 0 & 0 & -2
\end{array}\right)$
\end{matlabsymbolicoutput}
\begin{matlabsymbolicoutput}
C7 = 
    $\displaystyle \left(\begin{array}{cccc}
1 & 0 & 0 & 4\\
0 & 1 & 0 & -1\\
0 & 0 & 1 & -7\\
0 & 0 & 0 & 1
\end{array}\right)$
\end{matlabsymbolicoutput}
\begin{par}
\begin{flushleft}
a=4d     b=-1d     c=-7    d=0     dudo de esta respuesta
\end{flushleft}
\end{par}

\begin{par}
\begin{flushleft}
b)
\end{flushleft}
\end{par}

\begin{matlabsymbolicoutput}
ans = 
    $\displaystyle \left(\begin{array}{cc}
2 & 2\\
-6 & 2
\end{array}\right)$
\end{matlabsymbolicoutput}
\begin{matlabsymbolicoutput}
A = 
    $\displaystyle \left(\begin{array}{ccccc}
1 & -1 & 0 & 0 & 2\\
0 & 1 & -1 & 0 & 2\\
0 & 0 & 1 & -1 & -6\\
-1 & 0 & 0 & 1 & 2
\end{array}\right)$
\end{matlabsymbolicoutput}
\begin{matlabsymbolicoutput}
A1 = 
    $\displaystyle \left(\begin{array}{ccccc}
1 & -1 & 0 & 0 & 2\\
0 & 1 & -1 & 0 & 2\\
0 & 0 & 1 & -1 & -6\\
0 & -1 & 0 & 1 & 4
\end{array}\right)$
\end{matlabsymbolicoutput}
\begin{matlabsymbolicoutput}
A2 = 
    $\displaystyle \left(\begin{array}{ccccc}
1 & 0 & -1 & 0 & 4\\
0 & 1 & -1 & 0 & 2\\
0 & 0 & 1 & -1 & -6\\
0 & -1 & 0 & 1 & 4
\end{array}\right)$
\end{matlabsymbolicoutput}
\begin{matlabsymbolicoutput}
A3 = 
    $\displaystyle \left(\begin{array}{ccccc}
1 & 0 & -1 & 0 & 4\\
0 & 1 & -1 & 0 & 2\\
0 & 0 & 1 & -1 & -6\\
0 & 0 & -1 & 1 & 6
\end{array}\right)$
\end{matlabsymbolicoutput}
\begin{matlabsymbolicoutput}
A4 = 
    $\displaystyle \left(\begin{array}{ccccc}
1 & 0 & -1 & 0 & 4\\
0 & 1 & -1 & 0 & 2\\
0 & 0 & 1 & -1 & -6\\
0 & 0 & 0 & 0 & 0
\end{array}\right)$
\end{matlabsymbolicoutput}
\begin{matlabsymbolicoutput}
A5 = 
    $\displaystyle \left(\begin{array}{ccccc}
1 & 0 & -1 & 0 & 4\\
0 & 1 & 0 & -1 & -4\\
0 & 0 & 1 & -1 & -6\\
0 & 0 & 0 & 0 & 0
\end{array}\right)$
\end{matlabsymbolicoutput}
\begin{matlabsymbolicoutput}
A6 = 
    $\displaystyle \left(\begin{array}{ccccc}
1 & 0 & 0 & -1 & -2\\
0 & 1 & 0 & -1 & -4\\
0 & 0 & 1 & -1 & -6\\
0 & 0 & 0 & 0 & 0
\end{array}\right)$
\end{matlabsymbolicoutput}
\begin{par}
\begin{flushleft}
solucion alternativa
\end{flushleft}
\end{par}

\begin{matlabsymbolicoutput}
eq1 = 
    $\displaystyle a-b=2$
\end{matlabsymbolicoutput}
\begin{matlabsymbolicoutput}
eq2 = 
    $\displaystyle b-c=2$
\end{matlabsymbolicoutput}
\begin{matlabsymbolicoutput}
eq3 = 
    $\displaystyle c-d=-6$
\end{matlabsymbolicoutput}
\begin{matlabsymbolicoutput}
eq4 = 
    $\displaystyle d-a=2$
\end{matlabsymbolicoutput}
\begin{matlaboutput}
S = 
    a: [1x1 sym]
    b: [1x1 sym]
    c: [1x1 sym]
    d: [1x1 sym]

\end{matlaboutput}
\begin{matlabsymbolicoutput}
ans = 
    $\displaystyle -2$
\end{matlabsymbolicoutput}
\begin{matlabsymbolicoutput}
ans = 
    $\displaystyle -4$
\end{matlabsymbolicoutput}
\begin{matlabsymbolicoutput}
ans = 
    $\displaystyle -6$
\end{matlabsymbolicoutput}
\begin{matlabsymbolicoutput}
ans = 
    $\displaystyle 0$
\end{matlabsymbolicoutput}
\begin{par}
\begin{flushleft}
todo correcto
\end{flushleft}
\end{par}


\label{H_32D8FF25}
\matlabheadingthree{Ej 9}

\begin{par}
\begin{flushleft}
a)
\end{flushleft}
\end{par}

\begin{matlabsymbolicoutput}
C = 
    $\displaystyle \left(\begin{array}{ccc}
-12 & 2 & 11\\
0 & 6 & -10
\end{array}\right)$
\end{matlabsymbolicoutput}
\begin{par}
\begin{flushleft}
b)
\end{flushleft}
\end{par}

\begin{matlabsymbolicoutput}
A = 
    $\displaystyle \left(\begin{array}{cccc}
1 & -5 & 4 & 0\\
2 & 1 & 0 & 6
\end{array}\right)$
\end{matlabsymbolicoutput}
\begin{matlabsymbolicoutput}
ans = 
    $\displaystyle \left(\begin{array}{cc}
1 & 2\\
-5 & 1\\
4 & 0\\
0 & 6
\end{array}\right)$
\end{matlabsymbolicoutput}

\matlabheadingthree{Ej 10}

\begin{par}
\begin{flushleft}
a)
\end{flushleft}
\end{par}

\begin{matlabsymbolicoutput}
A = 
    $\displaystyle \left(\begin{array}{ccc}
-1 & 3 & 3\\
-2 & -1 & -4
\end{array}\right)$
\end{matlabsymbolicoutput}
\begin{par}
\begin{flushleft}
b)
\end{flushleft}
\end{par}

\begin{matlabsymbolicoutput}
A = 
    $\displaystyle \left(\begin{array}{cc}
2 & 0\\
1 & -1
\end{array}\right)$
\end{matlabsymbolicoutput}
\begin{par}
\begin{flushleft}
ps se armo
\end{flushleft}
\end{par}


\matlabheadingthree{Ej 11}

\begin{par}
\begin{flushleft}
demo que no se hacer porque no se como son las matrices simetricas
\end{flushleft}
\end{par}


\matlabheadingthree{Ej 12}

\begin{par}
\begin{flushleft}
11 pero de diagonales
\end{flushleft}
\end{par}


\matlabheadingthree{Ej 13}

\begin{par}
\begin{flushleft}
11
\end{flushleft}
\end{par}


\matlabheadingthree{Ej 14}

\begin{par}
\hfill \break
\end{par}

\begin{matlabsymbolicoutput}
A = 
    $\displaystyle \left(\begin{array}{ccc}
a_{1,1}  & a_{1,2}  & a_{1,3} \\
a_{2,1}  & a_{2,2}  & a_{2,3} \\
a_{3,1}  & a_{3,2}  & a_{3,3} 
\end{array}\right)$
\end{matlabsymbolicoutput}
\begin{par}
\begin{flushleft}
para que sean iguales tienen que ser iguales todos sus terminos
\end{flushleft}
\end{par}

\begin{par}
\begin{flushleft}
un ejemplo de sus terminos es aij=k*aji
\end{flushleft}
\end{par}

\begin{par}
\begin{flushleft}
\textbf{??}
\end{flushleft}
\end{par}


\matlabheadingthree{Ej 15}

\begin{par}
\begin{flushleft}
11
\end{flushleft}
\end{par}

\begin{par}
\begin{flushleft}
a)
\end{flushleft}
\end{par}

\begin{par}
\begin{flushleft}
b)
\end{flushleft}
\end{par}


\matlabheadingthree{Ej 16}

\begin{par}
\begin{flushleft}
a)
\end{flushleft}
\end{par}

\begin{matlabsymbolicoutput}
ans = 
    $\displaystyle \left(\begin{array}{cc}
2 & 2\\
0 & -2
\end{array}\right)$
\end{matlabsymbolicoutput}
\begin{par}
\begin{flushleft}
b)
\end{flushleft}
\end{par}

\begin{matlabsymbolicoutput}
ans = 
    $\displaystyle \left(\begin{array}{ccc}
-1 & -6 & -2\\
0 & 6 & 10
\end{array}\right)$
\end{matlabsymbolicoutput}
\begin{par}
\begin{flushleft}
c) \textbf{interesante}
\end{flushleft}
\end{par}

\begin{matlabsymbolicoutput}
C = 
    $\displaystyle \left(\begin{array}{ccc}
2\,a & 3\,b & c\\
5\,a & 7\,b & 4\,c
\end{array}\right)$
\end{matlabsymbolicoutput}
\begin{matlabsymbolicoutput}
c11 = 
    $\displaystyle 2\,a$
\end{matlabsymbolicoutput}
\begin{matlabsymbolicoutput}
c12 = 
    $\displaystyle 3\,b$
\end{matlabsymbolicoutput}
\begin{matlabsymbolicoutput}
c13 = 
    $\displaystyle c$
\end{matlabsymbolicoutput}
\begin{matlabsymbolicoutput}
c21 = 
    $\displaystyle 5\,a$
\end{matlabsymbolicoutput}
\begin{matlabsymbolicoutput}
c22 = 
    $\displaystyle 7\,b$
\end{matlabsymbolicoutput}
\begin{matlabsymbolicoutput}
c23 = 
    $\displaystyle 4\,c$
\end{matlabsymbolicoutput}
\begin{matlabsymbolicoutput}
C23 = 
    $\displaystyle 4\,c$
\end{matlabsymbolicoutput}

\matlabheadingthree{Ej 17 }

\begin{par}
\begin{flushleft}
despejamos la matriz que queremos el pedo radica en qeu se multiplican y no se como \textbf{dividir matrices}
\end{flushleft}
\end{par}

\begin{par}
\begin{flushleft}
a)
\end{flushleft}
\end{par}

\begin{matlaboutput}
A = 
       1              2       
       4             10       

\end{matlaboutput}
\begin{par}
\begin{flushleft}
b)
\end{flushleft}
\end{par}

\begin{matlaboutput}
A = 
      16/5           -3/5     
       2/5            9/5     

\end{matlaboutput}

\matlabheadingthree{Ej 18}


\matlabheadingthree{Ej 19}


\label{H_EFF10B42}
\matlabheadingthree{Ej 20}


\label{H_638BB5A6}
\matlabheadingthree{Ej 21}


\label{H_F94BA146}
\matlabheadingthree{Ej 22}


\label{H_13F8AC55}
\matlabheadingthree{Ej 23}


\label{H_94AA8155}
\matlabheadingthree{Ej 24}


\label{H_A8532C99}
\matlabheadingthree{Ej 25}


\label{H_34FB1CDD}
\matlabheadingthree{Ej 26}


\label{H_8C61C081}
\matlabheadingthree{Ej 27}


\label{H_DBB3B193}
\matlabheadingthree{Ej 28}


\label{H_55CB4F74}
\matlabheadingthree{Ej 29}


\label{H_8CF77302}
\matlabheadingthree{Ej 30}


\label{H_AD3BF2D3}
\matlabheadingthree{Ej 31}


\label{H_8D97FE59}
\matlabheadingthree{Ej 32}


\label{H_73258759}
\matlabheadingthree{Ej 33}


\label{H_15767BFA}
\matlabheadingthree{Ej 34}


\matlabheading{Ejercicios de Kolman-Hill}

\matlabheadingthree{1.1}

\begin{par}
\begin{flushleft}
1-14 son sistemas lineales por metodo de eliminacion gaussiana, no los hare
\end{flushleft}
\end{par}

\begin{par}
\begin{flushleft}
15 - 
\end{flushleft}
\end{par}

\begin{par}
\begin{flushleft}
16 - 
\end{flushleft}
\end{par}

\begin{par}
\begin{flushleft}
17 -
\end{flushleft}
\end{par}

\begin{par}
\begin{flushleft}
18 - entonces porque metodo?
\end{flushleft}
\end{par}

\begin{par}
\begin{flushleft}
19 - solo expandes mas la matriz 
\end{flushleft}
\end{par}

\begin{par}
\begin{flushleft}
20 - igual que la 19
\end{flushleft}
\end{par}

\begin{par}
\begin{flushleft}
21 - no hay figura 1.2
\end{flushleft}
\end{par}

\begin{par}
\begin{flushleft}
22 - a) 0 b) ? c) ? 
\end{flushleft}
\end{par}

\begin{par}
\begin{flushleft}
23 - mismo que el ultimo de la tarea 2
\end{flushleft}
\end{par}

\begin{par}
\begin{flushleft}
24 - 26 son del tipo del 23
\end{flushleft}
\end{par}

\begin{par}
\begin{flushleft}
27 - ya lo resolvi en la tarea 1
\end{flushleft}
\end{par}

\begin{par}
\begin{flushleft}
28 - 23
\end{flushleft}
\end{par}

\matlabheadingthree{Ejericios teoricos}

\begin{par}
\begin{flushleft}
T.1 - Demuestra que el sistema lineal que se obtiene al intercambiar dos ecuaciones en A tiene las mismas soluciones que A
\end{flushleft}
\end{par}

\begin{par}
\begin{flushleft}
T.2 - Demuestre que el sistema lineal obtenido al remplazar una ecuacion en A por un multiplo constante de la ecuacion diferente de cero, tiene exactamente las mismas soluciones que A
\end{flushleft}
\end{par}

\begin{par}
\begin{flushleft}
T.3 - Demuestre que el sistema lineal que se obtiene al remplazar una ecuacion en A por ella misma mas un multiplo de otra ecuacion en A tiene exactamente las mismas soluciones que A
\end{flushleft}
\end{par}

\begin{par}
\begin{flushleft}
T.4 - mismo que en tarea 1
\end{flushleft}
\end{par}

\matlabheadingthree{1.2}

\begin{par}
\begin{flushleft}
1- determinar valores especificos sin hacer toda la matriz
\end{flushleft}
\end{par}

\begin{par}
\begin{flushleft}
2 - sistema de ecuaciones sencillo presentado de forma medio rara
\end{flushleft}
\end{par}

\begin{par}
\begin{flushleft}
3 - 2
\end{flushleft}
\end{par}

\begin{par}
\begin{flushleft}
4 - 7 son operaciones de matrices sencillas
\end{flushleft}
\end{par}

\begin{par}
\begin{flushleft}
8 - combinacion lineal \textbf{?}
\end{flushleft}
\end{par}

\begin{par}
\begin{flushleft}
9 - 8 
\end{flushleft}
\end{par}

\begin{par}
\begin{flushleft}
10 - multiplicar una constante por una matriz de forma que te de otra, mas o menos se como hacerla pero vale la pena hacerla \textbf{ !!}
\end{flushleft}
\end{par}

\begin{par}
\begin{flushleft}
11 - 4-7
\end{flushleft}
\end{par}

\begin{par}
\begin{flushleft}
12 - 11
\end{flushleft}
\end{par}

\begin{par}
\begin{flushleft}
13 - Obtener una matriz a partir de una suma de matrices !!!
\end{flushleft}
\end{par}

\begin{par}
\begin{flushleft}
14-15 son 13 pero con vectores
\end{flushleft}
\end{par}

\matlabheadingthree{Ej teoricos}

\begin{par}
\begin{flushleft}
T.1 - Demuestra que la suma y la diferencia de dos matrices diagonales es una matriz diagonal
\end{flushleft}
\end{par}

\begin{par}
\begin{flushleft}
T.2 - Demuestre que la suma y la diferencia de dos matrices escalares es una matriz escalar
\end{flushleft}
\end{par}

\begin{par}
\begin{flushleft}
T.3 - 4-7 pero simbolico
\end{flushleft}
\end{par}

\begin{par}
\begin{flushleft}
T.4 - sea 0 la matriz de n x n tal que todas sus entradas son cero. Demuestre que si k es un numero real y a es una matriz de n x n tal qeu kA=0 entonces k=0 o A=0
\end{flushleft}
\end{par}

\begin{par}
\begin{flushleft}
T.5 -
\end{flushleft}
\end{par}

\begin{par}
\begin{flushleft}
a) demuestre que la suma y la diferencia de dos matrices triangulares superiores es una matriz triangular superior
\end{flushleft}
\end{par}

\begin{par}
\begin{flushleft}
b) demuestre lo que a) pero con inferiores
\end{flushleft}
\end{par}

\begin{par}
\begin{flushleft}
c) demo que si una matriz es inferior y superior al mismo tiempo entonces es diagnoal
\end{flushleft}
\end{par}

\begin{par}
\begin{flushleft}
T.6 - 
\end{flushleft}
\end{par}

\begin{par}
\begin{flushleft}
a) demo que si A es superior A\textasciicircum{}t es inferior
\end{flushleft}
\end{par}

\begin{par}
\begin{flushleft}
b) viceversa de a)
\end{flushleft}
\end{par}

\begin{par}
\begin{flushleft}
T.7 - cuales son las entradas de la diagonal principal de A-A\textasciicircum{}t si A es matriz cuadrada
\end{flushleft}
\end{par}

\begin{par}
\begin{flushleft}
T.8 - si x es un n vector demuestre que x+0=x
\end{flushleft}
\end{par}

\begin{par}
\begin{flushleft}
T.9 - haga una lista de todos los posibles 2 vectores binarios ¿cuantos hay?
\end{flushleft}
\end{par}

\begin{par}
\begin{flushleft}
T.10 - T.9 pero con 3 vector
\end{flushleft}
\end{par}

\begin{par}
\begin{flushleft}
T.11 - T.9 pero con 4 vector
\end{flushleft}
\end{par}

\matlabheadingthree{1.3}

\vspace{1em}

\matlabheadingthree{Ej teoricos}

\begin{par}
\begin{flushleft}
T.1 - sea x un n vector 
\end{flushleft}
\end{par}

\begin{par}
\begin{flushleft}
a) es posible que x punto x sea negativo?
\end{flushleft}
\end{par}

\begin{par}
\begin{flushleft}
b) si x punto x = 0, cual es el valor de x?
\end{flushleft}
\end{par}

\begin{par}
\begin{flushleft}
T.2 - 
\end{flushleft}
\end{par}

\begin{par}
\begin{flushleft}
a) demuestre que a punto b = b punto a
\end{flushleft}
\end{par}

\begin{par}
\begin{flushleft}
b) demuestre que (a + b) punto c = a punto c + b punto c
\end{flushleft}
\end{par}

\begin{par}
\begin{flushleft}
c) demuestre que (ka) punto b = a punto (kb) = k(a punto b)
\end{flushleft}
\end{par}

\begin{par}
\begin{flushleft}
T.3 - 
\end{flushleft}
\end{par}

\begin{par}
\begin{flushleft}
a) demuestre uqe si A tiene una dila de ceros, AB tiene una fila de ceros
\end{flushleft}
\end{par}

\begin{par}
\begin{flushleft}
b) demuestre que si B tienen una columna de ceros, AB tiene una columna de ceros
\end{flushleft}
\end{par}

\begin{par}
\begin{flushleft}
T.4 - 
\end{flushleft}
\end{par}

\begin{par}
\begin{flushleft}
Demuestre que el producto de dos matrices diagonales es una matriz diagnoal
\end{flushleft}
\end{par}

\begin{par}
\begin{flushleft}
T.5 - 
\end{flushleft}
\end{par}

\begin{par}
\begin{flushleft}
demuestre que el producto de dos matrices escalares es una matriz escalar
\end{flushleft}
\end{par}

\begin{par}
\begin{flushleft}
T.6 -
\end{flushleft}
\end{par}

\begin{par}
\begin{flushleft}
a) demuester que el producto de dos matrices superiores es una matriz superior
\end{flushleft}
\end{par}

\begin{par}
\begin{flushleft}
ya existe en algun otro lado
\end{flushleft}
\end{par}

\begin{par}
\begin{flushleft}
b) demuestre que el prducto de dos matrices inferiores es una matriz inferior
\end{flushleft}
\end{par}

\begin{par}
\begin{flushleft}
T.7 - sean A y B matrices diagonales de nxn ¿es cierto que AB = BA?
\end{flushleft}
\end{par}

\begin{par}
\begin{flushleft}
T.8 - 
\end{flushleft}
\end{par}

\begin{par}
\begin{flushleft}
a) sea a una matriz de 1xn y B una matriz de nxp demuestre que el producto de matrices aB puede escribirse como una combinacion lineal de las filas de B en los qeu los coeficientes cson las entradas de a
\end{flushleft}
\end{par}

\begin{par}
\begin{flushleft}
b) sean
\end{flushleft}
\end{par}

\begin{matlabsymbolicoutput}
ans = 
    $\displaystyle \left(\begin{array}{ccc}
1 & -2 & 3
\end{array}\right)$
\end{matlabsymbolicoutput}
\begin{matlabsymbolicoutput}
ans = 
    $\displaystyle \left(\begin{array}{ccc}
2 & 1 & -4\\
-3 & -2 & 3\\
4 & 5 & -2
\end{array}\right)$
\end{matlabsymbolicoutput}
\begin{par}
\begin{flushleft}
escriba aB como una combinacion lineal de las filas de B
\end{flushleft}
\end{par}

\begin{par}
\begin{flushleft}
T.9 - 
\end{flushleft}
\end{par}

\begin{par}
\begin{flushleft}
a) demuester que al jesima columna del producto de matrices AB es igual al producto de matrices A col\_j(B)
\end{flushleft}
\end{par}

\begin{par}
\begin{flushleft}
b) demuestre que el iesima fila del producto de matrices AB es igual al producto de matrices fil\_i(A)B
\end{flushleft}
\end{par}

\begin{par}
\begin{flushleft}
T.10 - sea A una matriz de mxn cuyas entradas son numeros reales, demuestre que si AA\textasciicircum{}t= 0 entonces A=0
\end{flushleft}
\end{par}

\begin{par}
\begin{flushleft}
T.11 - T.13 son de notacion sigma que ni al caso
\end{flushleft}
\end{par}

\begin{par}
\begin{flushleft}
T.14 - 
\end{flushleft}
\end{par}

\begin{par}
\begin{flushleft}
a) si u y v se consideran matrices de nx1 demuestre que u punto v = u\textasciicircum{}tv
\end{flushleft}
\end{par}

\begin{par}
\begin{flushleft}
b) lo mismo que a) pero u punto v = uv\textasciicircum{}t
\end{flushleft}
\end{par}

\begin{par}
\begin{flushleft}
c) si u se considerara una matriz de 1xn y v una matriz de nx1 demuestre que u punto v= uv
\end{flushleft}
\end{par}


\vspace{1em}

\label{H_A129AEC7}
\matlabheading{Funciones}

\end{document}
